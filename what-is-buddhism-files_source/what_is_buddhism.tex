\documentclass[12pt, openany]{book}

%PACKAGES%

\usepackage[outer=1in, inner=0.8in, top=0.7in, bottom=1in,a5paper]{geometry}
\usepackage{graphicx}

\usepackage[]{enumitem}
\setlist{noitemsep}
%\usepackage{letterspace}

\usepackage{fix-cm}%font size
\usepackage[unicode, hidelinks, pdfauthor={Ajahn Brahm}, pdftitle={What Is Buddhism?}, pdfsubject={Buddhism}, pdfkeywords={Buddhism, meditation,metta,love}, pdfproducer={LuaTeX beta-0.70.1}, pdfcreator={LaTeX2e}]{hyperref} %ADDS METADATA%



\setcounter{tocdepth}{0}

\usepackage{fontspec}

\setmainfont{Sarabun}

\newfontface\fancyfont[]{Charmonman-Regular}

\usepackage{polyglossia}
\setdefaultlanguage[numerals = thai]{thai}


\usepackage{setspace}

\setstretch{1.3}

\usepackage{sectsty}

\allsectionsfont{\mdseries}

\chapterfont{\raggedleft\fancyfont}

%\usepackage{pagegrid}

%PACKAGES%

%\pagegridsetup{top-left, step=3.435in}

%DRAW GRID%

%\usepackage{pagegrid}

%\pagegridsetup{top-left, step=0.2in}

%DRAW GRID%


%Linespace%
\usepackage{setspace}
%LINESPAce%

%HEADER%

%HEADER%

\usepackage{fancyhdr}

\setlength{\headheight}{15pt}
\pagestyle{fancy}
\fancyhf{}
\fancyfoot[CE,CO]{\footnotesize – \thepage \hspace{0.3em}–}
\renewcommand{\headrulewidth}{0pt}
\fancypagestyle{plain}{ %
\fancyhf{} % remove everything

\renewcommand{\headrulewidth}{0pt} 

\renewcommand{\footrulewidth}{0pt}}

%HEADER%

%HEADER%

%FONTS%



%FONTS%



\usepackage[]{microtype}

\frenchspacing


%TABLE OF CONTENTS%

\usepackage{tocloft}

\renewcommand*{\cfttoctitlefont}{\mdseries\Large}

\renewcommand\cftchappagefont{\mdseries}

\renewcommand{\cftchapfont}{\mdseries}

\renewcommand{\cftdotsep}{\cftnodots}



%TABLE OF CONTENTS%

%HANGING LEFT%


%HANGINGLEFT%

%WIDOWS & ORPHANS%

\widowpenalty=9000

\clubpenalty=9000

%WIDOWS & ORPHANS%



%DOCUMENT INFO. NOT USED IN TEXT.%

\title{What Is Buddhism?}

\author{Ajahn Brahm}

\date{}

%DOCUMENT INFO. NOT USED IN TEXT.%

\usepackage{anyfontsize}


\pagestyle{empty}
\begin{document}



\raggedbottom

\frontmatter

\begin{center}\end{center}
\begin{center}

\vspace{9em}

\fontsize{35.8272}{35.8272}{\fancyfont ศาสนาพุทธโดยสังเขป}

\vspace{9em}

{\Large พระอาจารย์พรหม}


\end{center}

\newpage

{\small

\noindent แจกเป็ นธรรมทานเท่านันห้ามจำหน่าย

\bigskip

\noindent What is Buddhism?
\noindent ศาสนาพุทธโดยสังเขป

\bigskip

\noindent สงวน{\wbr}ลิขสิทธิ์ ©พระ{\wbr}อาจารย์{\wbr}พรหม{\wbr}วัง{\wbr}โส เจ้าอาวาส วัด{\wbr}โพธิญาณ

\noindent ผู้นำ{\wbr}ทาง{\wbr}จิต{\wbr}วิญญาณ พุทธ{\wbr}สมาคม{\wbr}แห่ง{\wbr}รัฐ{\wbr}เวสเทิร์น{\wbr}ออสเตรเลีย

\bigskip

\noindent นคร{\wbr}เพิร์ท ประเทศ{\wbr}ออสเตรเลีย เดือน{\wbr}มกราคม พ.{\wbr}ศ. 2559

\bigskip

\noindent กรุณา{\wbr}ติดต่อ{\wbr}เจ้าของ{\wbr}ลิขสิทธิ์ 

\noindent หาก{\wbr}มี{\wbr}ความ{\wbr}ประสงค์{\wbr}จะ{\wbr}ดำเนิน{\wbr}การ{\wbr}จัด{\wbr}พิมพ์{\wbr}หนังสือ{\wbr}เล่ม{\wbr}นี้ 

\noindent ยกเว้น{\wbr}ใน{\wbr}กรณี{\wbr}ที่{\wbr}จะ{\wbr}พิมพ์{\wbr}เพื่อ{\wbr}แจก{\wbr}เป็น{\wbr}ธรรม{\wbr}ทาน{\wbr}เท่านั้น

}



\newpage


\tableofcontents


\chapter*{บทนำ}

เป็น{\wbr}เวลา{\wbr}กว่า 2,500 ปี{\wbr}แล้ว{\wbr}ที่{\wbr}ศาสนา{\wbr}ซึ่ง{\wbr}รู้จัก{\wbr}กัน{\wbr}ใน{\wbr}ปัจจุบัน{\wbr}ใน{\wbr}นาม “พระ{\wbr}พุทธ{\wbr}ศาสนา” ได้{\wbr}เป็น{\wbr}แรง{\wbr}บันดาล{\wbr}ใจ{\wbr}สำคัญ{\wbr}แก่{\wbr}อารยธรรม{\wbr}ที่{\wbr}รุ่งเรือง{\wbr}จำนวน{\wbr}มาก เป็น{\wbr}แหล่ง{\wbr}ที่{\wbr}มา{\wbr}แห่ง{\wbr}ความ{\wbr}สําเร็จ{\wbr}ทาง{\wbr}วัฒนธรรม{\wbr}ที่{\wbr}ยิ่งใหญ่ อีก{\wbr}ทั้ง เป็น{\wbr}ประทีป{\wbr}นำทาง{\wbr}ที่{\wbr}ยืนยง{\wbr}ทรง{\wbr}ความหมาย{\wbr}สํา{\wbr}หรับ{\wbr}ผู้คน{\wbr}นับ{\wbr}ล้าน{\wbr}ใน{\wbr}การ{\wbr}แสวง{\wbr}หา{\wbr}ซึ่ง{\wbr}จุด{\wbr}หมาย{\wbr}แห่ง{\wbr}ชีวิต  

ทุก{\wbr}วัน{\wbr}นี้ ชาย{\wbr}และ{\wbr}หญิง{\wbr}จําน{\wbr}วน{\wbr}มาก{\wbr}จาก{\wbr}ภูมิ{\wbr}หลัง{\wbr}ที่{\wbr}แตก{\wbr}ต่าง{\wbr}หลากหลาย{\wbr}ทั่ว{\wbr}โลก{\wbr}กํา{\wbr}ลัง{\wbr}ดำเนิน{\wbr}รอย{\wbr}ตาม{\wbr}คํา{\wbr}สอน{\wbr}ของ{\wbr}พระพุทธเจ้า 

แล้ว{\wbr}พระพุทธเจ้า{\wbr}ท่าน{\wbr}เป็น{\wbr}ใคร{\wbr}และ{\wbr}พระองค์{\wbr}ทรง{\wbr}สอน{\wbr}อะไร{\wbr}

\mainmatter

\pagestyle{fancy}


\chapter*{พระพุทธเจ้า}
\addcontentsline{toc}{chapter}{พระพุทธเจ้า}


บุรุษ{\wbr}ผู้{\wbr}ซึ่ง{\wbr}ต่อ{\wbr}มา{\wbr}ได้{\wbr}ตรัสรู้{\wbr}เป็น{\wbr}พระพุทธเจ้า{\wbr}ทรง{\wbr}ถือ{\wbr}กำเนิด{\wbr}เมื่อ{\wbr}ประมาณ 2,500 ปี{\wbr}ก่อน มี{\wbr}พระ{\wbr}นาม{\wbr}ว่า เจ้า{\wbr}ชาย{\wbr}สิทธัตถะ  พระองค์{\wbr}ทรง{\wbr}เป็น{\wbr}เจ้า{\wbr}ชาย{\wbr}แห่ง{\wbr}ราชสกุล{\wbr}โคตมะ{\wbr}ของ{\wbr}แคว้น{\wbr}เล็ก ๆ อยู่{\wbr}ใกล้{\wbr}กับ{\wbr}บริเวณ{\wbr}ที่{\wbr}ปัจจุบัน คือ พรมแดน{\wbr}ระหว่าง{\wbr}อินเดีย{\wbr}และ{\wbr}เนปาล  

แม้{\wbr}ว่า{\wbr}พระองค์{\wbr}จะ{\wbr}ทรง{\wbr}ได้{\wbr}รับ{\wbr}การ{\wbr}เลี้ยงดู{\wbr}ให้{\wbr}อยู่{\wbr}อย่าง{\wbr}เกษม{\wbr}สำราญ พรั่งพร้อม{\wbr}ด้วย{\wbr}ยศ{\wbr}ถา{\wbr}บรรดาศักดิ์{\wbr}อัน{\wbr}สูงส่ง แต่{\wbr}ความ{\wbr}เพลิดเพลิน{\wbr}ทาง{\wbr}วัตถุ{\wbr}เหล่า{\wbr}นั้น{\wbr}ก็{\wbr}ไม่{\wbr}อาจ{\wbr}ปิดบัง{\wbr}ชีวิต{\wbr}ที่{\wbr}ปราศจาก{\wbr}ความ{\wbr}สมบูรณ์{\wbr}จาก{\wbr}ชาย{\wbr}หนุ่ม{\wbr}ที่{\wbr}ช่าง{\wbr}สงสัย{\wbr}ใคร่{\wbr}รู้{\wbr}ต่าง{\wbr}จาก{\wbr}คน{\wbr}ทั่วไป  เมื่อ{\wbr}พระองค์{\wbr}ทรง{\wbr}มี{\wbr}พระ{\wbr}ชนมายุ 29 พรรษา จึง{\wbr}ทรง{\wbr}สละ{\wbr}ความ{\wbr}พรั่งพร้อม{\wbr}ใน{\wbr}ราชสมบัติ{\wbr}และ{\wbr}ครอบครัว{\wbr}เพื่อ{\wbr}แสวง{\wbr}หา{\wbr}ความหมาย{\wbr}ของ{\wbr}ชีวิต{\wbr}ที่{\wbr}ลึกซึ้ง{\wbr}กว่า{\wbr}ที่{\wbr}ทรง{\wbr}ประสบ โดย{\wbr}ทรง{\wbr}ใช้{\wbr}ชีวิต{\wbr}ใน{\wbr}ป่า{\wbr}เขา{\wbr}ที่{\wbr}ห่าง{\wbr}ไกล{\wbr}ผู้คน{\wbr}ทาง{\wbr}ตะวัน{\wbr}ออก{\wbr}เฉียง{\wbr}เหนือ{\wbr}ของ{\wbr}อินเดีย  

พระองค์{\wbr}ได้{\wbr}ทรง{\wbr}ศึกษา{\wbr}ใน{\wbr}สำนัก{\wbr}ของ{\wbr}นักบวช{\wbr}นัก{\wbr}ปรัชญา{\wbr}ที่{\wbr}มี{\wbr}ชื่อเสียง{\wbr}ที่สุด{\wbr}ทั้งหลาย{\wbr}ใน{\wbr}ยุค{\wbr}นั้น และ{\wbr}ได้{\wbr}ร่ำเรียน{\wbr}ทุก{\wbr}สิ่ง{\wbr}ทุก{\wbr}อย่าง{\wbr}จน{\wbr}หมด{\wbr}ภูมิปัญญา{\wbr}ของ{\wbr}อาจารย์{\wbr}ที่{\wbr}จะ{\wbr}สอน{\wbr}ให้ ได้ แต่{\wbr}คำ{\wbr}สอน{\wbr}ของ{\wbr}อาจารย์{\wbr}เหล่า{\wbr}นั้น{\wbr}ก็{\wbr}ยัง{\wbr}ไม่{\wbr}สามารถ{\wbr}ให้{\wbr}คำ{\wbr}ตอบ{\wbr}ที่{\wbr}พระองค์{\wbr}ทรง{\wbr}กำลัง{\wbr}แสวง{\wbr}หา{\wbr}ได้  หลัง{\wbr}จาก{\wbr}นั้น พระองค์{\wbr}จึง{\wbr}ทรง{\wbr}ตรากตรำ{\wbr}บำเพ็ญ{\wbr}ทุกรกิริยา{\wbr}อย่าง{\wbr}ยิ่งยวด แต่{\wbr}ก็{\wbr}ยัง{\wbr}ทรง{\wbr}ไม่{\wbr}พบ{\wbr}คำ{\wbr}ตอบ 

ใน{\wbr}วันเพ็ญ{\wbr}เดือน 6 เมื่อ{\wbr}ทรง{\wbr}พระ{\wbr}ชนมายุ{\wbr}ได้ 35 พรรษา พระองค์{\wbr}ทรง{\wbr}ประทับ{\wbr}นั่ง{\wbr}ที่{\wbr}โคน{\wbr}ต้นไม้{\wbr}ซึ่ง{\wbr}ปัจจุบัน{\wbr}เรียก{\wbr}ว่า ต้น{\wbr}โพธิ์ ใน{\wbr}ป่าละเมาะ{\wbr}อัน{\wbr}เงียบ{\wbr}สงบ{\wbr}ริม{\wbr}ฝั่ง{\wbr}แม่น้ำ{\wbr}เนรัญชรา และ{\wbr}ทรง{\wbr}สำรวม{\wbr}จิต{\wbr}เข้า{\wbr}สมาธิ{\wbr}ลึก{\wbr}จน{\wbr}เกิด{\wbr}ความ{\wbr}สงบ{\wbr}และ{\wbr}สว่าง  พระองค์{\wbr}ทรง{\wbr}ใช้{\wbr}จิต{\wbr}ที่{\wbr}สว่าง{\wbr}ผ่องใส{\wbr}ประกอบ{\wbr}ไป{\wbr}ด้วย{\wbr}พลานุภาพ{\wbr}อัน{\wbr}แจ่มแจ้ง{\wbr}จาก{\wbr}สภาวะ{\wbr}ความ{\wbr}สงบ{\wbr}นิ่ง{\wbr}ลึก{\wbr}ภายใน{\wbr}พิจารณา{\wbr}ความ{\wbr}จริง{\wbr}ของ{\wbr}จิต จักรวาล และ{\wbr}ชีวิต จน{\wbr}ตรัสรู้{\wbr}เป็น{\wbr}พระพุทธเจ้า  นับ{\wbr}แต่{\wbr}บัดนั้น{\wbr}เป็นต้น{\wbr}มา พระองค์{\wbr}ก็{\wbr}ทรง{\wbr}เป็น{\wbr}ที่{\wbr}รู้จัก{\wbr}ใน{\wbr}ฐานะ “สมเด็จ{\wbr}พระ{\wbr}สัมมา{\wbr}สัมพุทธ{\wbr}เจ้า ผู้{\wbr}รู้ ผู้{\wbr}ตื่น ผู้{\wbr}เบิกบาน” 

พระองค์{\wbr}ทรง{\wbr}เห็น{\wbr}แจ้ง{\wbr}ซึ่ง{\wbr}สัจธรรม{\wbr}อัน{\wbr}ลึกซึ้ง{\wbr}ถึง{\wbr}ธรรมชาติ{\wbr}ของ{\wbr}จิต{\wbr}และ{\wbr}ปรากฏการณ์{\wbr}ทั้งหลาย{\wbr}ทั้งมวล  การ{\wbr}รู้แจ้ง{\wbr}นี้{\wbr}มิ{\wbr}ได้{\wbr}มา{\wbr}จาก{\wbr}การ{\wbr}เปิดเผย{\wbr}ของ{\wbr}เทพเจ้า{\wbr}ใด ๆ แต่{\wbr}เป็น{\wbr}การ{\wbr}ค้น{\wbr}พบ{\wbr}ด้วย{\wbr}ตัว{\wbr}พระองค์{\wbr}เอง{\wbr}โดย{\wbr}การ{\wbr}เข้า{\wbr}สมาธิ{\wbr}ถึง{\wbr}ขั้น{\wbr}ที่{\wbr}ลึก{\wbr}ที่สุด{\wbr}และ{\wbr}เข้าถึง{\wbr}จิต{\wbr}ที่{\wbr}ใส{\wbr}สว่าง  นั่น{\wbr}หมายความ{\wbr}ว่า พระองค์{\wbr}ทรง{\wbr}เป็น{\wbr}อิสระ{\wbr}จาก{\wbr}บ่วง{\wbr}ของ{\wbr}ความ{\wbr}โลภ ความ{\wbr}โกรธ และ{\wbr}ความ{\wbr}หลง ตลอด{\wbr}จน{\wbr}ความ{\wbr}ทุกข์{\wbr}ที่{\wbr}อยู่{\wbr}ภายใน{\wbr}ทั้งมวล  และ{\wbr}พระองค์{\wbr}ก็{\wbr}ได้{\wbr}ทรง{\wbr}เข้าถึง{\wbr}ความ{\wbr}สงบ{\wbr}สันติ{\wbr}ที่ ไม่{\wbr}สั่น{\wbr}คลอน

\chapter*{คำ{\wbr}สอน{\wbr}ของ{\wbr}พระพุทธเจ้า}
\addcontentsline{toc}{chapter}{คำ{\wbr}สอน{\wbr}ของ{\wbr}พระพุทธเจ้า}


เมื่อ{\wbr}พระพุทธเจ้า{\wbr}ทรง{\wbr}แจ่มแจ้ง{\wbr}ถึง{\wbr}เป้าหมาย คือ การ{\wbr}ตรัสรู้{\wbr}แล้ว พระองค์{\wbr}ทรง{\wbr}ใช้{\wbr}เวลา 45 ปี{\wbr}หลัง{\wbr}จาก{\wbr}นั้น{\wbr}สั่งสอน{\wbr}เส้นทาง{\wbr}ที่{\wbr}ไม่{\wbr}ว่า{\wbr}จะ{\wbr}เป็น{\wbr}ใคร จะ{\wbr}มี{\wbr}เชื้อชาติ ชน{\wbr}ชั้น{\wbr}หรือ{\wbr}เพศ{\wbr}ใด ๆ ก็{\wbr}ตาม หาก{\wbr}เมื่อ{\wbr}ได้{\wbr}ลงมือ{\wbr}ปฏิบัติ{\wbr}ตาม{\wbr}ด้วย{\wbr}ความ{\wbr}เพียร{\wbr}แล้ว ย่อม{\wbr}จะ{\wbr}ไป{\wbr}ถึง{\wbr}จุด{\wbr}หมาย{\wbr}แห่ง{\wbr}การ{\wbr}ตื่น{\wbr}รู้{\wbr}เฉก{\wbr}เช่น{\wbr}เดียวกัน  คําสอน{\wbr}เกี่ยว{\wbr}กับ{\wbr}เส้นทาง{\wbr}สาย{\wbr}นี้{\wbr}เรียก{\wbr}ว่า “ธรรมะ” ซึ่ง{\wbr}มี{\wbr}ความหมาย{\wbr}ตรง{\wbr}ตัว{\wbr}คือ ธรรมชาติ{\wbr}ของ{\wbr}สรรพ{\wbr}สิ่ง{\wbr}หรือ{\wbr}ความ{\wbr}จริง{\wbr}อัน{\wbr}เป็น{\wbr}เหตุ{\wbr}เบื้องหลัง{\wbr}แห่ง{\wbr}การ{\wbr}ดำรง{\wbr}อยู่{\wbr}ของ{\wbr}สิ่ง{\wbr}ต่าง ๆ  อย่างไร{\wbr}ก็{\wbr}ตาม เนื้อหา{\wbr}คำ{\wbr}สอน{\wbr}เหล่า{\wbr}นั้น{\wbr}มี{\wbr}มาก{\wbr}เกิน{\wbr}จะ{\wbr}กล่าว{\wbr}ได้{\wbr}หมด{\wbr}ใน{\wbr}หนังสือ{\wbr}เล่ม{\wbr}เล็ก ๆ เล่ม{\wbr}นี้ แต่{\wbr}อาตมา{\wbr}จะ{\wbr}ให้{\wbr}ภาพ{\wbr}รวม{\wbr}คำ{\wbr}สอน{\wbr}ของ{\wbr}พระพุทธเจ้า{\wbr}โดย{\wbr}แยก{\wbr}ออก{\wbr}เป็น{\wbr}เจ็ด{\wbr}หัวข้อ{\wbr}ดัง{\wbr}ต่อ{\wbr}ไป{\wbr}นี้{\wbr}

\section{วิถี{\wbr}แห่ง{\wbr}การ{\wbr}ไต่ถาม{\wbr}หา{\wbr}ความ{\wbr}จริง }


พระพุทธเจ้า{\wbr}ทรง{\wbr}เตือน{\wbr}อย่าง{\wbr}มาก{\wbr}ไม่{\wbr}ให้{\wbr}เรา{\wbr}หลับ{\wbr}หู{\wbr}หลับ{\wbr}ตา{\wbr}เชื่อ และ ทรง{\wbr}สนับสนุน{\wbr}ให้{\wbr}เรา{\wbr}ใช้{\wbr}วิถี{\wbr}แห่ง{\wbr}การ{\wbr}ไต่ถาม{\wbr}หา{\wbr}ความ{\wbr}จริง{\wbr}อย่าง{\wbr}จริงใจ  พระองค์{\wbr}ทรง{\wbr}ชี้{\wbr}ให้{\wbr}เห็น{\wbr}ถึง{\wbr}โทษ{\wbr}ของ{\wbr}การ{\wbr}เชื่อ{\wbr}โดย{\wbr}อาศัย{\wbr}เพียง{\wbr}เหตุ{\wbr}ดัง{\wbr}ต่อ{\wbr}ไป{\wbr}นี้ คือ เป็น{\wbr}เรื่อง{\wbr}ที่{\wbr}ได้ยิน{\wbr}ได้{\wbr}ฟัง{\wbr}มา เป็น{\wbr}ประเพณี{\wbr}ปฏิบัติ{\wbr}กัน{\wbr}มา ใคร ๆ ก็{\wbr}บอก{\wbr}อย่าง{\wbr}นั้น คัมภีร์{\wbr}โบราณ{\wbr}เขียน{\wbr}ไว้ เป็น{\wbr}วาจา{\wbr}ของ{\wbr}สิ่ง{\wbr}เหนือ{\wbr}ธรรมชาติ หรือ{\wbr}เพราะ{\wbr}เรา{\wbr}มี{\wbr}ศรัทธา{\wbr}ใน{\wbr}ครู{\wbr}บา{\wbr}อาจารย์ ผู้{\wbr}อาวุโส{\wbr}หรือ{\wbr}นักบวช  แต่{\wbr}ทรง{\wbr}สอน{\wbr}ให้{\wbr}เรา{\wbr}เปิด{\wbr}ใจ{\wbr}กว้าง{\wbr}และ{\wbr}พิจารณา{\wbr}ตรวจสอบ{\wbr}อย่าง{\wbr}ถี่ถ้วน{\wbr}เพื่อ{\wbr}หา{\wbr}ความ{\wbr}จริง{\wbr}จาก{\wbr}ประสบการณ์{\wbr}ชีวิต{\wbr}ของ{\wbr}ตัว{\wbr}เรา{\wbr}เอง  เมื่อ{\wbr}เรา{\wbr}มอง{\wbr}เห็นด้วย{\wbr}ตัวเอง{\wbr}ว่า ความคิด{\wbr}เห็น{\wbr}ใด{\wbr}สอดคล้อง{\wbr}กับ{\wbr}ทั้ง{\wbr}ประสบการณ์{\wbr}และ{\wbr}เหตุผล อีก{\wbr}ทั้ง ยัง{\wbr}นํา{\wbr}ไป{\wbr}สู่{\wbr}ความ{\wbr}สุข{\wbr}ของ{\wbr}ทั้ง{\wbr}ตัว{\wbr}เรา{\wbr}และ{\wbr}ผู้{\wbr}อื่น เมื่อนั้น เรา{\wbr}ค่อย{\wbr}ยอมรับ{\wbr}ความคิด{\wbr}เห็น{\wbr}ดัง{\wbr}กล่าว{\wbr}และ{\wbr}ดํา{\wbr}เนิน{\wbr}ชีวิต{\wbr}ไป{\wbr}ตาม{\wbr}นั้น{\wbr}

หลักการ{\wbr}นี้{\wbr}ยัง{\wbr}ใช้{\wbr}ได้{\wbr}กับ{\wbr}คํา{\wbr}สอน{\wbr}ของ{\wbr}พระพุทธเจ้า{\wbr}อีก{\wbr}ด้วย  เรา{\wbr}ควร{\wbr}พิจารณา{\wbr}ไตร่ตรอง{\wbr}และ{\wbr}ตรวจสอบ{\wbr}โดย{\wbr}ใช้{\wbr}จิต{\wbr}ที่{\wbr}ผ่องใส{\wbr}ซึ่ง{\wbr}เกิด{\wbr}จาก{\wbr}สมาธิ  เมื่อ{\wbr}สมาธิ{\wbr}ของ{\wbr}เรา{\wbr}ลึก{\wbr}ขึ้น เรา{\wbr}ก็{\wbr}จะ{\wbr}เห็น{\wbr}แจ้ง{\wbr}ถึง{\wbr}คํา{\wbr}สอน{\wbr}เหล่า{\wbr}นี้{\wbr}ได้{\wbr}ด้วย{\wbr}ปัญญา{\wbr}ของ{\wbr}ตนเอง และ{\wbr}เมื่อนั้น{\wbr}คํา{\wbr}สอน{\wbr}ของ{\wbr}พระพุทธเจ้า{\wbr}จัก{\wbr}กลาย{\wbr}เป็น{\wbr}สัจธรรม{\wbr}ประจำ{\wbr}ใจ{\wbr}เรา{\wbr}เอง{\wbr}ซึ่ง{\wbr}นำ{\wbr}ไป{\wbr}สู่{\wbr}การ{\wbr}หลุดพ้น{\wbr}อัน{\wbr}เป็น{\wbr}สุข{\wbr}ยิ่ง{\wbr}

ผู้{\wbr}เดินทาง{\wbr}บน{\wbr}ถนน{\wbr}สาย{\wbr}แห่ง{\wbr}การ{\wbr}ตรวจสอบ{\wbr}นี้{\wbr}จำเป็น{\wbr}ที่{\wbr}จะ{\wbr}ต้อง{\wbr}เปิด{\wbr}ใจ{\wbr}ให้{\wbr}กว้าง  การ{\wbr}เปิด{\wbr}ใจ{\wbr}กว้าง{\wbr}นี้{\wbr}ไม่{\wbr}ได้{\wbr}หมายความ{\wbr}ว่า{\wbr}เรา{\wbr}จะ{\wbr}ต้อง{\wbr}ยอมรับ{\wbr}ทุก{\wbr}แนว{\wbr}ความคิด{\wbr}หรือ{\wbr}ทรรศนะ  แต่{\wbr}หมายความ{\wbr}ว่า{\wbr}เรา{\wbr}จะ{\wbr}ไม่{\wbr}ขุ่น{\wbr}เคือง{\wbr}กับ{\wbr}สิ่ง{\wbr}ที่{\wbr}เรา{\wbr}ไม่{\wbr}สามารถ{\wbr}ยอมรับ{\wbr}ได้ เพราะ{\wbr}ใน{\wbr}ระหว่าง{\wbr}การ{\wbr}เดินทาง เรา{\wbr}อาจ{\wbr}พบ{\wbr}ว่า สิ่ง{\wbr}ที่{\wbr}เรา{\wbr}ไม่{\wbr}เห็นด้วย{\wbr}ใน{\wbr}ตอน{\wbr}แรก{\wbr}นั้น{\wbr}อาจ{\wbr}เป็น{\wbr}จริง{\wbr}ได้  

ดังนั้น หาก{\wbr}พร้อม{\wbr}ที่{\wbr}จะ{\wbr}ไตร่ตรอง{\wbr}ตรวจสอบ{\wbr}ด้วย{\wbr}ใจ{\wbr}ที่{\wbr}เปิด{\wbr}กว้าง{\wbr}กัน{\wbr}แล้ว ขอ{\wbr}เชิญ{\wbr}อ่าน{\wbr}คํา{\wbr}สอน{\wbr}พื้นฐาน{\wbr}ของ{\wbr}พระพุทธเจ้า{\wbr}

\section{อริยสัจ 4}


คําสอน{\wbr}หลัก{\wbr}ของ{\wbr}พระพุทธเจ้า{\wbr}ไม่{\wbr}ได้{\wbr}มุ่ง{\wbr}เน้น{\wbr}ไป{\wbr}ที่{\wbr}การ{\wbr}คาด{\wbr}เดา{\wbr}ทาง{\wbr}ปรัชญา{\wbr}เกี่ยว{\wbr}กับ{\wbr}พระ{\wbr}ผู้{\wbr}สร้าง กําเนิดของ{\wbr}จักรวาล หรือ{\wbr}การ{\wbr}เข้าถึง{\wbr}โลก{\wbr}สวรรค์{\wbr}นิรันดร  คําสอน{\wbr}ของ{\wbr}พระองค์{\wbr}เน้น{\wbr}ที่{\wbr}ความ{\wbr}จริง{\wbr}แท้{\wbr}แห่ง{\wbr}ความ{\wbr}ทุกข์{\wbr}ของ{\wbr}มนุษย์{\wbr}เรา{\wbr}และ{\wbr}ความ{\wbr}จํา{\wbr}เป็น{\wbr}เร่งด่วน{\wbr}ใน{\wbr}การ{\wbr}ค้น{\wbr}หา{\wbr}วิธี{\wbr}ดับ{\wbr}ทุกข์{\wbr}อย่าง{\wbr}ถาวร{\wbr}เพื่อ{\wbr}ให้{\wbr}พ้น{\wbr}จาก{\wbr}ความ{\wbr}ไม่{\wbr}สงบสุข{\wbr}ทั้งมวล  พระพุทธเจ้า{\wbr}ทรง{\wbr}อุปมา{\wbr}กับ{\wbr}ชาย{\wbr}คน{\wbr}หนึ่ง{\wbr}ที่{\wbr}ถูก{\wbr}ยิง{\wbr}ด้วย{\wbr}ลูกศร{\wbr}อาบ{\wbr}ยาพิษ ซึ่ง{\wbr}ก่อน{\wbr}ที่{\wbr}เขา{\wbr}จะ{\wbr}ยอม{\wbr}ให้{\wbr}แพทย์{\wbr}รักษา เขา{\wbr}กลับ{\wbr}ขอ{\wbr}รู้{\wbr}ให้{\wbr}ได้{\wbr}ก่อน{\wbr}ว่า{\wbr}ผู้{\wbr}ใด{\wbr}เป็น{\wbr}คน{\wbr}ยิง มี{\wbr}สถานะ{\wbr}หรือ{\wbr}ชน{\wbr}ชั้น{\wbr}ใด{\wbr}ใน{\wbr}สังคม มา{\wbr}จาก{\wbr}ไหน ใช้{\wbr}ธนู{\wbr}แบบ{\wbr}ไหน ลูกศร{\wbr}ทํา{\wbr}มา{\wbr}จาก{\wbr}อะไร  ชาย{\wbr}ผู้{\wbr}โง่เขลา{\wbr}นี้{\wbr}จะ{\wbr}ต้อง{\wbr}ตาย{\wbr}ก่อน{\wbr}ที่{\wbr}จะ{\wbr}ได้{\wbr}รับ{\wbr}คํา{\wbr}ตอบ{\wbr}อย่าง{\wbr}แน่นอน{\wbr}

ใน{\wbr}ทํา{\wbr}นอง{\wbr}เดียวกัน พระพุทธเจ้า{\wbr}ตรัส{\wbr}ว่า ความ{\wbr}จํา{\wbr}เป็น{\wbr}เร่งด่วน{\wbr}ที่สุด{\wbr}ของ{\wbr}เรา{\wbr}คือ การ{\wbr}ค้น{\wbr}หา{\wbr}วิธีการ{\wbr}ที่{\wbr}ถาวร{\wbr}ใน{\wbr}การ{\wbr}ปลดเปลื้อง{\wbr}ความ{\wbr}ไม่{\wbr}พอใจ{\wbr}ที่{\wbr}เกิด{\wbr}ขึ้น{\wbr}ซ้ำ{\wbr}แล้ว{\wbr}ซ้ำ{\wbr}เล่า เพราะ{\wbr}มัน{\wbr}ได้{\wbr}ขโมย{\wbr}ความ{\wbr}สุข{\wbr}ของ{\wbr}เรา{\wbr}ไป{\wbr}และ{\wbr}ปล่อย{\wbr}ให้{\wbr}เรา{\wbr}ตก{\wbr}อยู่{\wbr}ใน{\wbr}ห้วง{\wbr}ความ{\wbr}ขัดแย้ง  สมมติฐาน{\wbr}เชิง{\wbr}ปรัชญา{\wbr}มี{\wbr}ความ{\wbr}สําคัญ{\wbr}รอง{\wbr}ลง{\wbr}มา แต่{\wbr}จะ{\wbr}ให้{\wbr}ดี{\wbr}ที่สุด ควร{\wbr}จะ{\wbr}ทิ้ง{\wbr}เอา{\wbr}ไว้{\wbr}ก่อน{\wbr}จน{\wbr}กว่า{\wbr}จะ{\wbr}ได้{\wbr}ฝึก{\wbr}จิต{\wbr}ฝึก{\wbr}ใจ{\wbr}ใน{\wbr}การ{\wbr}ทํา{\wbr}สมาธิ{\wbr}จน{\wbr}ถึง{\wbr}ขั้น{\wbr}ที่{\wbr}สามารถ{\wbr}ตรวจสอบ{\wbr}ความ{\wbr}เป็น{\wbr}จริง{\wbr}ได้{\wbr}อย่าง{\wbr}ชัดเจน{\wbr}และ{\wbr}เห็น{\wbr}ความ{\wbr}จริง{\wbr}ได้{\wbr}ด้วย{\wbr}ตนเอง{\wbr}

ดังนั้น คําสอน{\wbr}หลัก{\wbr}ของ{\wbr}พระพุทธเจ้า{\wbr}ซึ่ง{\wbr}เป็น{\wbr}รากฐาน{\wbr}คํา{\wbr}สอน{\wbr}อื่น ๆ ของ{\wbr}พระองค์ คือ อริยสัจ 4 หรือ ความ{\wbr}จริง{\wbr}อัน{\wbr}ประเสริฐ 4 ประการ อัน{\wbr}ได้แก่{\wbr}

\newpage

\begin{enumerate}[noitemsep]

\item สิ่ง{\wbr}มี{\wbr}ชีวิต{\wbr}ทั้งหลาย{\wbr}ซึ่ง{\wbr}รวม{\wbr}ถึง{\wbr}มนุษย์{\wbr}และ{\wbr}สรรพ{\wbr}สัตว์{\wbr}ทั้งหลาย{\wbr}ต่าง{\wbr}ต้อง{\wbr}ประสบ{\wbr}ความ{\wbr}ทุกข์{\wbr}จาก{\wbr}ความ{\wbr}ผิดหวัง ความ{\wbr}โศกเศร้า ความ{\wbr}รู้สึก{\wbr}ไม่{\wbr}สบาย{\wbr}กาย{\wbr}ไม่{\wbr}สบายใจ ความ{\wbr}วิตก{\wbr}กังวล ฯลฯ ใน{\wbr}รูปแบบ{\wbr}ต่าง ๆ   กล่าว{\wbr}โดย{\wbr}สรุป{\wbr}คือ สิ่ง{\wbr}มี{\wbr}ชีวิต{\wbr}ทั้งหลาย{\wbr}ล้วน{\wbr}ต้อง{\wbr}ประสบ{\wbr}ความ{\wbr}ทุกข์{\wbr}

\item สาเหตุ{\wbr}ของ{\wbr}ความ{\wbr}ทุกข์{\wbr}นี้ คือ ความ{\wbr}ทะยาน{\wbr}อยาก{\wbr}ที่{\wbr}เกิด{\wbr}จาก{\wbr}ความ{\wbr}หลง{\wbr}ใน{\wbr}มายา{\wbr}ว่า{\wbr}มี “ดวงวิญญาณ”  (ดู{\wbr}หัวข้อ{\wbr}ที่ 7 ด้าน{\wbr}ล่าง)

\item เมื่อ{\wbr}บรรลุ{\wbr}มรรคผล{\wbr}นิพพาน{\wbr}ก็{\wbr}จะ{\wbr}พ้น{\wbr}ทุกข์ ซึ่ง{\wbr}เป็น{\wbr}ผล{\wbr}จาก{\wbr}การ{\wbr}ปล่อย{\wbr}วาง{\wbr}ความ{\wbr}หลง{\wbr}ใน{\wbr}มายา{\wbr}ว่า{\wbr}มี "ดวง{\wbr}วิญญาณ" และ{\wbr}เป็น{\wbr}ผล{\wbr}ให้{\wbr}ตัณหา{\wbr}และ{\wbr}โทสะ{\wbr}หมด{\wbr}สิ้น{\wbr}ไป 

\item สันติสุข{\wbr}และ{\wbr}ความ{\wbr}รู้{\wbr}ตื่น{\wbr}เบิกบาน{\wbr}นี้{\wbr}สามารถ{\wbr}เข้าถึง{\wbr}ได้{\wbr}โดย{\wbr}การ{\wbr}ฝึกฝน{\wbr}อย่าง{\wbr}ค่อย{\wbr}เป็น{\wbr}ค่อย{\wbr}ไป ตาม{\wbr}แนวทาง{\wbr}ที่{\wbr}เรียก{\wbr}ว่า ทาง{\wbr}สาย{\wbr}กลาง หรือ มรรค{\wbr}มี{\wbr}องค์ 8

    \end{enumerate}

จะ{\wbr}เป็น{\wbr}การ{\wbr}เข้าใจ{\wbr}ผิด{\wbr}อย่าง{\wbr}ยิ่ง{\wbr}ที่{\wbr}จะ{\wbr}มอง{\wbr}ว่า{\wbr}คำ{\wbr}สอน{\wbr}นี้ "มอง{\wbr}โลก{\wbr}ใน{\wbr}แง่{\wbr}ร้าย" ด้วย{\wbr}เหตุ{\wbr}ว่า{\wbr}เป็น{\wbr}คำ{\wbr}สอน{\wbr}ที่{\wbr}เริ่มต้น{\wbr}ด้วย{\wbr}การ{\wbr}มุ่ง{\wbr}เน้น{\wbr}ไป{\wbr}ที่{\wbr}ความ{\wbr}ทุกข์  

อัน{\wbr}ที่{\wbr}จริง พุทธ{\wbr}ศาสนา “มอง{\wbr}โลก{\wbr}ตาม{\wbr}ความ{\wbr}เป็น{\wbr}จริง” เพราะ{\wbr}กล้า{\wbr}เผชิญ{\wbr}กับ{\wbr}ความ{\wbr}จริง{\wbr}ที่{\wbr}ว่า{\wbr}ชีวิต{\wbr}มี{\wbr}ทุกข์{\wbr}หลากหลาย{\wbr}ประการ และ{\wbr}พุทธ{\wbr}ศาสนา{\wbr}ยัง “มอง{\wbr}โลก{\wbr}ใน{\wbr}แง่{\wbr}ดี” โดย{\wbr}แสดง{\wbr}ให้{\wbr}เห็น{\wbr}ถึง{\wbr}จุด{\wbr}สิ้นสุด{\wbr}ของ{\wbr}ปัญหา คือ{\wbr}พระ{\wbr}นิพพาน หรือ{\wbr}ความ{\wbr}รู้{\wbr}ตื่น{\wbr}ใน{\wbr}ชีวิต{\wbr}นี้  

บุคคล{\wbr}ที่{\wbr}บรรลุ{\wbr}ถึง{\wbr}สันติสุข{\wbr}อย่าง{\wbr}แท้{\wbr}จริง{\wbr}เหล่า{\wbr}นั้น{\wbr}เป็น{\wbr}ตัวอย่าง{\wbr}ที่{\wbr}ได้{\wbr}สร้าง{\wbr}แรง{\wbr}บันดาล{\wbr}ใจ{\wbr}ให้{\wbr}เห็น{\wbr}อย่าง{\wbr}ถ่องแท้{\wbr}ว่า พุทธ{\wbr}ศาสนา{\wbr}มิ{\wbr}ได้{\wbr}มอง{\wbr}โลก{\wbr}ใน{\wbr}แง่{\wbr}ร้าย{\wbr}เลย หาก{\wbr}แต่{\wbr}เป็น{\wbr}เส้นทาง{\wbr}อัน{\wbr}นำ{\wbr}ไป{\wbr}สู่{\wbr}ความ{\wbr}สุข{\wbr}ที่{\wbr}แท้{\wbr}จริง{\wbr}

\section{ทาง{\wbr}สาย{\wbr}กลาง หรือ มรรค{\wbr}มี{\wbr}องค์ 8}


เส้นทาง{\wbr}ที่{\wbr}จะ{\wbr}ไป{\wbr}ถึง{\wbr}การ{\wbr}พ้น{\wbr}จาก{\wbr}ทุกข์{\wbr}ทั้งมวล{\wbr}ได้{\wbr}นั้น{\wbr}เรียก{\wbr}ว่า ทาง{\wbr}สาย{\wbr}กลาง เนื่อง{\wbr}จาก{\wbr}เป็น{\wbr}เส้นทาง{\wbr}ที่{\wbr}หลีก{\wbr}เลี่ยง{\wbr}การ{\wbr}กระทำ{\wbr}สุด{\wbr}โต่ง{\wbr}ทั้ง{\wbr}สอง{\wbr}ประการ คือ การ{\wbr}หมกมุ่น{\wbr}อยู่{\wbr}ใน{\wbr}กาม{\wbr}และ{\wbr}การ{\wbr}ทรมาน{\wbr}ตนเอง  ต่อ{\wbr}เมื่อ{\wbr}กาย{\wbr}มี{\wbr}ความ{\wbr}สบาย{\wbr}ตาม{\wbr}สมควร{\wbr}โดย{\wbr}จิต{\wbr}ไม่{\wbr}มัวเมา{\wbr}ใน{\wbr}ความ{\wbr}สะดวก{\wbr}สบาย{\wbr}เท่านั้น จิต{\wbr}จึง{\wbr}จะ{\wbr}มี{\wbr}ความ{\wbr}ผ่องใส{\wbr}และ{\wbr}พลัง{\wbr}ที่{\wbr}จะ{\wbr}เข้า{\wbr}สมาธิ{\wbr}ได้{\wbr}ลึก{\wbr}และ{\wbr}ค้น{\wbr}พบ{\wbr}ความ{\wbr}จริง  ทาง{\wbr}สาย{\wbr}กลาง{\wbr}นี้{\wbr}ประกอบ{\wbr}ด้วย{\wbr}การ{\wbr}เจริญ{\wbr}ซึ่ง{\wbr}ศีล สมาธิ และ{\wbr}ปัญญา ด้วย{\wbr}ความ{\wbr}เพียร ซึ่ง{\wbr}จะ{\wbr}อธิบาย{\wbr}อย่าง{\wbr}ละเอียด{\wbr}ต่อ{\wbr}ไป ใน{\wbr}เรื่อง{\wbr}ของ{\wbr}มรรค{\wbr}มี{\wbr}องค์ 8 อัน{\wbr}ประกอบ{\wbr}ด้วย{\wbr}

\begin{enumerate}

\item สัมมา{\wbr}ทิฏฐิ ความเห็น{\wbr}ชอบ{\wbr}ตาม{\wbr}ทำนอง{\wbr}คลอง{\wbr}ธรรม{\wbr}

\item สัมมา{\wbr}สังกัปปะ ความ{\wbr}ดำริ{\wbr}ใน{\wbr}ทาง{\wbr}ที่{\wbr}ชอบ{\wbr}

\item สัมมา{\wbr}วาจา วาจา{\wbr}ชอบ (ประพฤติ{\wbr}วจี{\wbr}สุจริต)

\item สัมมา{\wbr}กัมมันตะ การ{\wbr}ปฏิบัติ{\wbr}ชอบ (ประพฤติ{\wbr}กาย{\wbr}สุจริต)

\item สัมมา{\wbr}อาชีวะ การ{\wbr}เลี้ยง{\wbr}ชีวิต{\wbr}ใน{\wbr}ทาง{\wbr}ที่{\wbr}ชอบ{\wbr}

\item สัมมา{\wbr}วายามะ ความ{\wbr}เพียร{\wbr}ชอบ{\wbr}

\item สัมมา{\wbr}สติ สติ{\wbr}หรือ{\wbr}ความ{\wbr}ระลึก{\wbr}ชอบ{\wbr}

\item สัมมา{\wbr}สมาธิ สมาธิ{\wbr}ชอบ{\wbr}

\end{enumerate}

({\wbr}คำ{\wbr}ว่า “ชอบ” ดัง{\wbr}กล่าว{\wbr}ข้าง{\wbr}ต้น{\wbr}แปล{\wbr}มา{\wbr}จาก{\wbr}คำ{\wbr}บาลี{\wbr}ว่า “สัมมา” ซึ่ง{\wbr}แปล{\wbr}ว่า ถูกต้อง ใน{\wbr}ที่{\wbr}นี้ หมาย{\wbr}ถึง เอื้อ{\wbr}ต่อ{\wbr}การ{\wbr}ให้{\wbr}เกิด{\wbr}ความ{\wbr}สุข{\wbr}และ{\wbr}การ{\wbr}รู้แจ้ง{\wbr}เห็น{\wbr}จริง)

สัมมา{\wbr}วาจา{\wbr}หรือ{\wbr}การ{\wbr}เจรจา{\wbr}ชอบ สัมมา{\wbr}กัมมันตะ{\wbr}หรือ{\wbr}การ{\wbr}ปฏิบัติ{\wbr}ชอบ และ{\wbr}สัมมา{\wbr}อาชีวะ{\wbr}หรือ{\wbr}การ{\wbr}เลี้ยง{\wbr}ชีวิต{\wbr}ใน{\wbr}ทาง{\wbr}ที่{\wbr}ชอบ คือ การ{\wbr}ฝึก{\wbr}ตน{\wbr}ให้{\wbr}มี{\wbr}คุณ{\wbr}งาม{\wbr}ความ{\wbr}ดี{\wbr}หรือ{\wbr}ศีลธรรม  สำหรับ{\wbr}ฆราวาส{\wbr}ชาว{\wbr}พุทธ{\wbr}ที่{\wbr}กำลัง{\wbr}ฝึกฝน{\wbr}ตนเอง{\wbr}ใน{\wbr}ข้อ{\wbr}นี้ นั่น{\wbr}ก็{\wbr}คือ การ{\wbr}ถือ{\wbr}ศีล 5 โดย{\wbr}การ{\wbr}งดเว้น{\wbr}จาก{\wbr}สิ่ง{\wbr}ต่อ{\wbr}ไป{\wbr}นี้ 

\begin{enumerate}

\item จงใจ{\wbr}ฆ่า{\wbr}สิ่ง{\wbr}มี{\wbr}ชีวิต{\wbr}ใด ๆ

\item การ{\wbr}ถือ{\wbr}เอา{\wbr}สิ่งของ{\wbr}ที่{\wbr}เจ้าของ{\wbr}ไม่{\wbr}ได้{\wbr}ให้{\wbr}

\item ประพฤติ{\wbr}ผิด{\wbr}ใน{\wbr}กาม โดย{\wbr}เฉพาะ{\wbr}การ{\wbr}คบ{\wbr}ชู้{\wbr}

\item กล่าว{\wbr}เท็จ{\wbr}

\item ดื่ม{\wbr}เครื่อง{\wbr}ดอง{\wbr}ของ{\wbr}เมา และ{\wbr}เสพ{\wbr}ยา{\wbr}หรือ{\wbr}สาร{\wbr}เสพติด{\wbr}ที่{\wbr}มิ{\wbr}ใช่{\wbr}เพื่อ{\wbr}การ{\wbr}แพทย์{\wbr}อัน{\wbr}จะ{\wbr}ทำให้{\wbr}ความ{\wbr}มี{\wbr}สติ{\wbr}และ{\wbr}ความ{\wbr}สามารถ{\wbr}ใน{\wbr}การ{\wbr}ตัดสิน{\wbr}ใจ{\wbr}บน{\wbr}พื้นฐาน{\wbr}ทาง{\wbr}ศีลธรรม{\wbr}ลด{\wbr}ลง{\wbr}

\end{enumerate}

สัมมา{\wbr}วายามะ{\wbr}หรือ{\wbr}ความ{\wbr}เพียร{\wbr}ชอบ สัมมา{\wbr}สติ{\wbr}หรือ{\wbr}ความ{\wbr}ระลึก{\wbr}ชอบ และ{\wbr}สัมมา{\wbr}สมาธิ{\wbr}หรือ{\wbr}สมาธิ{\wbr}ชอบ เป็น{\wbr}องค์{\wbr}ประกอบ{\wbr}ของ{\wbr}การ{\wbr}ทำ{\wbr}สมาธิ อัน{\wbr}จะ{\wbr}ทำให้{\wbr}จิต{\wbr}บริสุทธิ์{\wbr}จาก{\wbr}สภาวะ{\wbr}ความ{\wbr}สุข{\wbr}ที่{\wbr}เกิด{\wbr}จาก{\wbr}ความ{\wbr}สงบ{\wbr}นิ่ง{\wbr}ภายใน อีก{\wbr}ทั้ง ยัง{\wbr}ทำให้{\wbr}จิต{\wbr}มี{\wbr}พลัง{\wbr}สามารถ{\wbr}เห็น{\wbr}แจ้ง{\wbr}ถึง{\wbr}ความหมาย{\wbr}ของ{\wbr}ชีวิต{\wbr}ด้วย{\wbr}ปัญญา{\wbr}อัน{\wbr}ลึกซึ้ง{\wbr}

สัมมา{\wbr}ทิฏฐิ{\wbr}หรือ{\wbr}ความเห็น{\wbr}ชอบ{\wbr}ตาม{\wbr}ทำนอง{\wbr}คลอง{\wbr}ธรรม{\wbr}และ{\wbr}สัมมา{\wbr}สังกัปปะ{\wbr}หรือ{\wbr}ความ{\wbr}ดำริ{\wbr}ใน{\wbr}ทาง{\wbr}ที่{\wbr}ชอบ คือ การ{\wbr}เกิด{\wbr}พุทธิ{\wbr}ปัญญา{\wbr}ที่{\wbr}จะ{\wbr}นำ{\wbr}ไป{\wbr}สู่{\wbr}การ{\wbr}พ้น{\wbr}ทุกข์{\wbr}และ{\wbr}การ{\wbr}เปลี่ยนแปลง{\wbr}ของ{\wbr}ตัว{\wbr}เรา{\wbr}อย่าง{\wbr}สิ้นเชิง  รวม{\wbr}ทั้ง{\wbr}ยัง{\wbr}ทำให้{\wbr}เกิด{\wbr}ความ{\wbr}สงบ{\wbr}สันติ{\wbr}อัน{\wbr}ไม่{\wbr}สั่น{\wbr}คลอน{\wbr}และ{\wbr}ความ{\wbr}เมตตา{\wbr}กรุณา{\wbr}อัน{\wbr}ไม่{\wbr}ย่อหย่อน  

พระพุทธองค์{\wbr}ทรง{\wbr}กล่าว{\wbr}ว่า หาก{\wbr}ปราศจาก{\wbr}ซึ่ง{\wbr}การ{\wbr}ขัด{\wbr}เกลา{\wbr}ศีลธรรม{\wbr}ให้{\wbr}บริสุทธิ์{\wbr}แล้ว จะ{\wbr}ไม่{\wbr}มี{\wbr}ทาง{\wbr}บรรลุ{\wbr}ถึง{\wbr}สมาธิ{\wbr}ที่{\wbr}แท้{\wbr}จริง  และ{\wbr}หาก{\wbr}ปราศจาก{\wbr}ซึ่ง{\wbr}สมาธิ{\wbr}ที่{\wbr}แท้{\wbr}จริง จะ{\wbr}ไม่{\wbr}มี{\wbr}ทาง{\wbr}ที่{\wbr}จะ{\wbr}รู้แจ้ง{\wbr}เห็น{\wbr}จริง{\wbr}ได้{\wbr}เลย  

ด้วย{\wbr}เหตุ{\wbr}นี้ เส้นทาง{\wbr}ปฏิบัติ{\wbr}ตาม{\wbr}คำ{\wbr}สอน{\wbr}ของ{\wbr}พระพุทธองค์ จึง{\wbr}เป็น{\wbr}เส้นทาง{\wbr}เดิน{\wbr}ที่{\wbr}ต้อง{\wbr}ฝึก{\wbr}ไป{\wbr}เป็น{\wbr}ลำดับ เป็น{\wbr}เส้นทาง{\wbr}สาย{\wbr}กลาง{\wbr}ที่{\wbr}กอปร{\wbr}ด้วย{\wbr}ศีล สมาธิ และ{\wbr}ปัญญา ซึ่ง{\wbr}ได้{\wbr}อธิบาย{\wbr}ไว้{\wbr}ใน{\wbr}เรื่อง{\wbr}มรรค{\wbr}มี{\wbr}องค์ 8 และ{\wbr}จะ{\wbr}นำพา{\wbr}ไป{\wbr}สู่{\wbr}ความ{\wbr}สุข{\wbr}และ{\wbr}การ{\wbr}หลุดพ้น{\wbr}

\section{กรรม}


“กรรม” หมาย{\wbr}ถึง “การ{\wbr}กระทำ{\wbr}ด้วย{\wbr}เจตนา”  ตาม{\wbr}หลัก{\wbr}กฎ{\wbr}แห่ง{\wbr}กรรม การ{\wbr}กระทำ{\wbr}โดย{\wbr}เจตนา{\wbr}จะ{\wbr}ส่ง{\wbr}ผล{\wbr}ตาม{\wbr}มา{\wbr}อย่าง{\wbr}ไม่{\wbr}มี{\wbr}ทาง{\wbr}หลีก{\wbr}เลี่ยง{\wbr}ได้  

การ{\wbr}กระทำ{\wbr}ด้วย{\wbr}กาย วาจา และ{\wbr}ใจ ที่{\wbr}นำ{\wbr}ความ{\wbr}ทุกข์{\wbr}มา{\wbr}สู่{\wbr}ตนเอง ผู้{\wbr}อื่น หรือ{\wbr}ทั้ง{\wbr}ตนเอง{\wbr}และ{\wbr}ผู้{\wbr}อื่น{\wbr}นั้น{\wbr}เรียก{\wbr}ว่า “อกุศล{\wbr}กรรม” หรือ “กรรม{\wbr}ชั่ว”  อกุศล{\wbr}กรรม{\wbr}นี้{\wbr}มี{\wbr}แรงผลัก{\wbr}ดัน{\wbr}มา{\wbr}จาก{\wbr}ความ{\wbr}โลภ ความ{\wbr}โกรธ และ{\wbr}ความ{\wbr}หลง  เนื่อง{\wbr}จาก{\wbr}อกุศล{\wbr}กรรม{\wbr}นำ{\wbr}มา{\wbr}ซึ่ง{\wbr}ผลลัพธ์{\wbr}ที่{\wbr}เป็น{\wbr}ทุกข์ เรา{\wbr}จึง{\wbr}ไม่{\wbr}ควร{\wbr}สร้าง{\wbr}อกุศล{\wbr}กรรม{\wbr}

ยัง{\wbr}มี{\wbr}การ{\wbr}กระทำ{\wbr}ด้วย{\wbr}กาย วาจา และ{\wbr}ใจ ที่{\wbr}นำ{\wbr}ความ{\wbr}สุข{\wbr}ความ{\wbr}เจริญ{\wbr}มา{\wbr}สู่{\wbr}ตนเอง ผู้{\wbr}อื่น หรือ{\wbr}ทั้ง{\wbr}ตนเอง{\wbr}และ{\wbr}ผู้{\wbr}อื่น เรียก{\wbr}ว่า “กุศล{\wbr}กรรม” หรือ “กรรม{\wbr}ดี”  ซึ่ง{\wbr}มี{\wbr}แรง{\wbr}จูง{\wbr}ใจ{\wbr}มา{\wbr}จาก{\wbr}ความ{\wbr}เอื้อ{\wbr}อารี ความ{\wbr}เมตตา{\wbr}กรุณา หรือ{\wbr}ปัญญา และ{\wbr}เพราะ{\wbr}กุศล{\wbr}กรรม{\wbr}นำ{\wbr}มา{\wbr}ซึ่ง{\wbr}ผลลัพธ์{\wbr}ที่{\wbr}ดี{\wbr}งาม เรา{\wbr}จึง{\wbr}ควร{\wbr}สร้าง{\wbr}กุศล{\wbr}กรรม{\wbr}ให้{\wbr}บ่อย{\wbr}ที่สุด{\wbr}เท่า{\wbr}ที่{\wbr}จะ{\wbr}ทำ{\wbr}ได้{\wbr}

เรา{\wbr}สามารถ{\wbr}ที่{\wbr}จะ{\wbr}ได้{\wbr}รับ{\wbr}ผล{\wbr}กรรม ณ ตอน{\wbr}นี้{\wbr}เลย เช่น เวลา{\wbr}ที่{\wbr}เรา{\wbr}แสดงออก{\wbr}ถึง{\wbr}ความ{\wbr}มี{\wbr}น้ำใจ หรือ{\wbr}แค่{\wbr}คิด{\wbr}ใน{\wbr}แง่{\wbr}ดี เรา{\wbr}มัก{\wbr}จะ{\wbr}รู้สึก{\wbr}พอใจ{\wbr}และ{\wbr}มี{\wbr}ความ{\wbr}สุข แต่{\wbr}เวลา{\wbr}ที่{\wbr}เรา{\wbr}แสดงออก{\wbr}ถึง{\wbr}ความ{\wbr}ไร้{\wbr}น้ำใจ เรา{\wbr}จะ{\wbr}รู้สึก{\wbr}ว่า มี{\wbr}ความ{\wbr}สุข{\wbr}น้อย{\wbr}ลง พลังงาน{\wbr}ทาง{\wbr}จิตใจ{\wbr}เหือดหาย และ{\wbr}สติ{\wbr}ก็{\wbr}ถดถอย  เมื่อ{\wbr}เรา{\wbr}หมั่น{\wbr}สังเกต{\wbr}ว่า การ{\wbr}กระทำ{\wbr}ที่{\wbr}เรา{\wbr}ตั้งใจ{\wbr}ทำ{\wbr}มี{\wbr}ผล{\wbr}ต่อ{\wbr}ตัว{\wbr}เรา{\wbr}เอง{\wbr}อย่างไร เรา{\wbr}ก็{\wbr}จะ{\wbr}เริ่ม{\wbr}เข้าใจ{\wbr}ถึง{\wbr}การ{\wbr}ทำงาน{\wbr}ของ{\wbr}กฎ{\wbr}แห่ง{\wbr}กรรม และ{\wbr}ผล{\wbr}ที่{\wbr}ตาม{\wbr}มา{\wbr}คือ เรา{\wbr}เกิด{\wbr}แรง{\wbr}จูง{\wbr}ใจ{\wbr}อย่าง{\wbr}มหาศาล{\wbr}ที่{\wbr}จะ{\wbr}ดำเนิน{\wbr}ชีวิต{\wbr}ให้{\wbr}อยู่{\wbr}ตาม{\wbr}ทำนอง{\wbr}คลอง{\wbr}ธรรม{\wbr}มาก{\wbr}ขึ้น{\wbr}

พระพุทธเจ้า{\wbr}ทรง{\wbr}สอน{\wbr}ว่า ไม่{\wbr}มี{\wbr}สรรพ{\wbr}สัตว์{\wbr}ใด ๆ ไม่{\wbr}ว่า{\wbr}จะ{\wbr}เป็น{\wbr}เทพเจ้า{\wbr}หรือ{\wbr}ใคร{\wbr}ก็{\wbr}ตาม ล้วน{\wbr}ไม่{\wbr}มี{\wbr}อำนาจ{\wbr}ที่{\wbr}จะ{\wbr}หยุด{\wbr}ผล{\wbr}แห่ง{\wbr}กุศล{\wbr}กรรม{\wbr}และ{\wbr}อกุศล{\wbr}กรรม{\wbr}ได้  ความ{\wbr}จริง{\wbr}ที่{\wbr}ว่า เรา{\wbr}ต่าง{\wbr}ต้อง{\wbr}รับ{\wbr}ผล{\wbr}แห่ง{\wbr}การ{\wbr}กระทำ{\wbr}ของ{\wbr}เรา เป็น{\wbr}แรง{\wbr}จูง{\wbr}ใจ{\wbr}ที่{\wbr}ทรง{\wbr}พลัง{\wbr}สำหรับ{\wbr}ชาว{\wbr}พุทธ ที่{\wbr}จะ{\wbr}หลีก{\wbr}เลี่ยง{\wbr}การ{\wbr}ทำ{\wbr}กรรม{\wbr}ชั่ว{\wbr}ใน{\wbr}ทุก{\wbr}รูปแบบ และ{\wbr}ทำ{\wbr}กรรม{\wbr}ดี{\wbr}ให้{\wbr}มาก{\wbr}ที่สุด{\wbr}เท่า{\wbr}ที่{\wbr}จะ{\wbr}ทำ{\wbr}ได้{\wbr}

แม้{\wbr}ว่า{\wbr}เรา{\wbr}จะ{\wbr}ไม่{\wbr}อาจ{\wbr}หลีก{\wbr}หนี{\wbr}ผล{\wbr}แห่ง{\wbr}กรรม{\wbr}ชั่ว แต่{\wbr}เรา{\wbr}สามารถ{\wbr}บรรเทา{\wbr}ความ{\wbr}รุนแรง{\wbr}ได้  เมื่อ{\wbr}เรา{\wbr}ใส่{\wbr}เกลือ{\wbr}หนึ่ง{\wbr}ช้อน{\wbr}ลง{\wbr}ใน{\wbr}แก้ว น้ำ{\wbr}ทั้ง{\wbr}แก้ว{\wbr}จะ{\wbr}เค็ม{\wbr}มาก แต่{\wbr}หาก{\wbr}เรา{\wbr}นำ{\wbr}เกลือ{\wbr}หนึ่ง{\wbr}ช้อน{\wbr}นี้{\wbr}ไป{\wbr}ใส่{\wbr}ใน{\wbr}ทะเลสาบ{\wbr}น้ำ{\wbr}จืด รสชาติ{\wbr}ของ{\wbr}น้ำ{\wbr}ใน{\wbr}ทะเลสาบ{\wbr}แทบ{\wbr}จะ{\wbr}ไม่{\wbr}เปลี่ยน{\wbr}ไป{\wbr}เลย  โดย{\wbr}นัย{\wbr}เดียวกัน ผล{\wbr}แห่ง{\wbr}กรรม{\wbr}ชั่ว{\wbr}ของ{\wbr}บุคคล{\wbr}ที่{\wbr}ปกติ{\wbr}ทำ{\wbr}กรรม{\wbr}ดี{\wbr}เพียง{\wbr}เล็กน้อย ก็{\wbr}ย่อม{\wbr}เป็น{\wbr}ความ{\wbr}ทุกข์{\wbr}หนักหนา{\wbr}สาหัส แต่{\wbr}ผล{\wbr}แห่ง{\wbr}กรรม{\wbr}ชั่ว{\wbr}เดียวกัน{\wbr}นี้ หาก{\wbr}กระทำ{\wbr}โดย{\wbr}บุคคล{\wbr}ที่{\wbr}หมั่น{\wbr}ทำ{\wbr}กรรม{\wbr}ดี{\wbr}เป็น{\wbr}ประจำ ก็{\wbr}ย่อม{\wbr}แสดง{\wbr}ผล{\wbr}เพียง{\wbr}เบาบาง{\wbr}เท่านั้น 

ดังนั้น กฎ{\wbr}แห่ง{\wbr}กรรม{\wbr}ตาม{\wbr}ธรรมชาติ{\wbr}จึง{\wbr}เป็น{\wbr}พลัง{\wbr}ผลักดัน{\wbr}และ{\wbr}เหตุผล{\wbr}สำหรับ{\wbr}ชาว{\wbr}พุทธ{\wbr}ใน{\wbr}การ{\wbr}ประพฤติ{\wbr}ปฏิบัติ{\wbr}ตาม{\wbr}ศีลธรรม{\wbr}จรรยา{\wbr}และ{\wbr}มี{\wbr}จิต{\wbr}เมตตา{\wbr}กรุณา{\wbr}ใน{\wbr}สังคม{\wbr}ของ{\wbr}เรา 

\section{การ{\wbr}เกิด{\wbr}ใหม่{\wbr}หรือ{\wbr}การ{\wbr}เวียน{\wbr}ว่าย{\wbr}ตาย{\wbr}เกิด}


พระพุทธเจ้า{\wbr}ทรง{\wbr}ระลึก{\wbr}ชาติ{\wbr}ใน{\wbr}ปาง{\wbr}ก่อน ๆ ของ{\wbr}พระองค์{\wbr}ได้  แม้{\wbr}ใน{\wbr}ปัจจุบัน ก็{\wbr}ยัง{\wbr}มี{\wbr}ภิกษุ{\wbr}ภิกษุณี และ{\wbr}ฆราวาส{\wbr}ที่{\wbr}ระลึก{\wbr}ชาติ{\wbr}ได้{\wbr}เช่น{\wbr}กัน  ความ{\wbr}ทรง{\wbr}จำ{\wbr}ที่{\wbr}แจ่ม{\wbr}ชัด{\wbr}นี้{\wbr}เป็น{\wbr}ผล{\wbr}มา{\wbr}จาก{\wbr}การ{\wbr}มี{\wbr}สมาธิ{\wbr}ที่{\wbr}ลึก  ผู้{\wbr}ที่{\wbr}สามารถ{\wbr}ระลึก{\wbr}ชาติ{\wbr}ก่อน ๆ ของ{\wbr}ตน{\wbr}ได้{\wbr}ก็{\wbr}จะ{\wbr}เห็น{\wbr}ความ{\wbr}จริง{\wbr}ของ{\wbr}การ{\wbr}เวียน{\wbr}ว่าย{\wbr}ตาย{\wbr}เกิด{\wbr}ซึ่ง{\wbr}จะ{\wbr}ทำให้{\wbr}มอง{\wbr}ชีวิต{\wbr}ใน{\wbr}ชาติ{\wbr}นี้{\wbr}จาก{\wbr}มุม{\wbr}ที่{\wbr}มี{\wbr}ความหมาย  

การ{\wbr}จะ{\wbr}เข้าใจ{\wbr}กฎ{\wbr}แห่ง{\wbr}กรรม{\wbr}ได้{\wbr}นั้น{\wbr}จะ{\wbr}ต้อง{\wbr}มอง{\wbr}ภาย{\wbr}ใต้{\wbr}กรอบ{\wbr}ที่{\wbr}ว่า{\wbr}ชีวิต{\wbr}ได้{\wbr}เวียน{\wbr}ว่าย{\wbr}ตาย{\wbr}เกิด{\wbr}มา{\wbr}หลาย{\wbr}ภพ{\wbr}หลาย{\wbr}ชาติ เพราะ{\wbr}บาง{\wbr}ครั้ง{\wbr}กว่า{\wbr}ผล{\wbr}กรรม{\wbr}จะ{\wbr}ปรากฏ ก็{\wbr}ใช้{\wbr}เวลา{\wbr}นาน{\wbr}หลาย{\wbr}ชาติ  ดังนั้น กรรม{\wbr}และ{\wbr}การ{\wbr}เวียน{\wbr}ว่าย{\wbr}ตาย{\wbr}เกิด{\wbr}จึง{\wbr}เป็น{\wbr}ตัว{\wbr}อธิบาย{\wbr}ถึง{\wbr}เหตุผล{\wbr}ที่{\wbr}เรา{\wbr}เกิด{\wbr}มา{\wbr}ไม่{\wbr}เท่าเทียม{\wbr}กัน{\wbr}อย่าง{\wbr}เห็น{\wbr}ได้{\wbr}ชัด เช่น ทำไม{\wbr}บาง{\wbr}คน{\wbr}เกิด{\wbr}มา{\wbr}บน{\wbr}กอง{\wbr}เงิน{\wbr}กอง{\wbr}ทอง ใน{\wbr}ขณะ{\wbr}ที่{\wbr}บาง{\wbr}คน{\wbr}เกิด{\wbr}มา{\wbr}ท่ามกลาง{\wbr}ความ{\wbr}ยากไร้{\wbr}แสน{\wbr}เข็ญ  ทำไม{\wbr}เด็ก{\wbr}บาง{\wbr}คน{\wbr}เกิด{\wbr}มา{\wbr}สุขภาพ{\wbr}ดี มี{\wbr}อวัยวะ{\wbr}ครบ 32 ใน{\wbr}ขณะ{\wbr}ที่{\wbr}บาง{\wbr}คน{\wbr}เกิด{\wbr}มา{\wbr}พิการ{\wbr}หรือ{\wbr}มี{\wbr}โรค{\wbr}ประจำ{\wbr}ตัว  

อย่างไร{\wbr}ก็{\wbr}ดี เรา{\wbr}ไม่{\wbr}ควร{\wbr}มอง{\wbr}ผล{\wbr}แห่ง{\wbr}อกุศล{\wbr}กรรม{\wbr}ที่{\wbr}สร้าง{\wbr}ความ{\wbr}ทุกข์{\wbr}ยาก{\wbr}ให้{\wbr}เรา{\wbr}นั้น{\wbr}ว่า{\wbr}เป็น{\wbr}การ{\wbr}ลงโทษ{\wbr}การ{\wbr}กระทำ{\wbr}ที่{\wbr}ไม่{\wbr}ดี แต่{\wbr}ควร{\wbr}มอง{\wbr}ว่า กฎ{\wbr}แห่ง{\wbr}กรรม{\wbr}นี้{\wbr}เป็น{\wbr}กฎ{\wbr}อัน{\wbr}เป็น{\wbr}ไป{\wbr}ตาม{\wbr}หลัก{\wbr}แห่ง{\wbr}ธรรมชาติ  และ{\wbr}จาก{\wbr}การ{\wbr}ที่{\wbr}เรา{\wbr}ประสบ{\wbr}ผล{\wbr}แห่ง{\wbr}กรรม{\wbr}ใน{\wbr}ชาติ{\wbr}นี้ ทำให้{\wbr}เรา{\wbr}ได้{\wbr}เรียนรู้{\wbr}ถึง{\wbr}พลัง{\wbr}แห่ง{\wbr}เมตตา{\wbr}

การ{\wbr}เวียน{\wbr}ว่าย{\wbr}ตาย{\wbr}เกิด{\wbr}มิ{\wbr}ได้{\wbr}วนเวียน{\wbr}อยู่{\wbr}แต่{\wbr}ใน{\wbr}มนุษย์{\wbr}ภูมิ{\wbr}เท่านั้น  พระพุทธเจ้า{\wbr}ทรง{\wbr}ชี้{\wbr}ให้{\wbr}เห็น{\wbr}ว่า มนุษย์{\wbr}ภูมิ{\wbr}นั้น{\wbr}เป็น{\wbr}เพียง{\wbr}หนึ่ง{\wbr}ใน{\wbr}หลากหลาย{\wbr}ภพ{\wbr}ภูมิ ซึ่ง{\wbr}ได้แก่ สวรรค์{\wbr}ชั้น{\wbr}ต่าง ๆ นรก{\wbr}ภูมิ เดรัจฉาน และ{\wbr}เปรต  ไม่{\wbr}ใช่{\wbr}เพียง{\wbr}ว่า{\wbr}เรา{\wbr}อาจ{\wbr}ไป{\wbr}เกิด{\wbr}ใน{\wbr}ภพ{\wbr}ภูมิ{\wbr}เหล่า{\wbr}นี้{\wbr}ใน{\wbr}ชาติ{\wbr}หน้า{\wbr}ได้{\wbr}เท่านั้น แต่{\wbr}เรา{\wbr}อาจ{\wbr}จะ{\wbr}เกิด{\wbr}มา{\wbr}จาก{\wbr}ภพ{\wbr}ภูมิ{\wbr}ใด ๆ ที่{\wbr}กล่าว{\wbr}มา{\wbr}แล้ว{\wbr}นี้{\wbr}ก็{\wbr}ได้  ข้อเท็จจริง{\wbr}นี้{\wbr}จึง{\wbr}สามารถ{\wbr}อธิบาย{\wbr}ข้อ{\wbr}โต้แย้ง{\wbr}เรื่อง{\wbr}การ{\wbr}เกิด{\wbr}ใหม่{\wbr}ที่{\wbr}ว่า “ถ้า{\wbr}การ{\wbr}เวียน{\wbr}ว่าย{\wbr}ตาย{\wbr}เกิด{\wbr}เป็น{\wbr}เรื่อง{\wbr}จริง ทำไม{\wbr}ทุก{\wbr}วัน{\wbr}นี้ เรา{\wbr}จึง{\wbr}มี{\wbr}มนุษย์{\wbr}มาก{\wbr}กว่า{\wbr}เมื่อ{\wbr}ร้อย{\wbr}ปี{\wbr}ก่อน{\wbr}เป็น 10 เท่า”  คำ{\wbr}ตอบ{\wbr}คือ คน{\wbr}ที่{\wbr}มี{\wbr}ชีวิต{\wbr}อยู่{\wbr}ใน{\wbr}ปัจจุบัน{\wbr}นี้{\wbr}มา{\wbr}จาก{\wbr}หลากหลาย{\wbr}ภพ{\wbr}ภูมิ{\wbr}นั่นเอง{\wbr}

เมื่อ{\wbr}เรา{\wbr}เข้าใจ{\wbr}ว่า ตัว{\wbr}เรา{\wbr}เอง{\wbr}ก็{\wbr}วนเวียน{\wbr}เกิด{\wbr}และ{\wbr}ตาย{\wbr}ใน{\wbr}ภพ{\wbr}ภูมิ{\wbr}ต่าง ๆ เรา{\wbr}จะ{\wbr}ให้{\wbr}เกียรติ{\wbr}และ{\wbr}เมตตา{\wbr}สรรพ{\wbr}สัตว์{\wbr}ทั้งหลาย{\wbr}ใน{\wbr}ภพ{\wbr}ภูมิ{\wbr}เหล่า{\wbr}นี้{\wbr}มาก{\wbr}ขึ้น  ยก{\wbr}ตัวอย่าง{\wbr}เช่น เรา{\wbr}จะ{\wbr}ไม่{\wbr}อยาก{\wbr}เอาเปรียบ{\wbr}สัตว์{\wbr}เมื่อ{\wbr}เห็น{\wbr}ว่า สรรพ{\wbr}สัตว์{\wbr}ทั้งหลาย{\wbr}กับ{\wbr}เรา{\wbr}เกี่ยวโยง{\wbr}กัน{\wbr}ใน{\wbr}วัฏจักร{\wbr}แห่ง{\wbr}การ{\wbr}เวียน{\wbr}ว่าย{\wbr}ตาย{\wbr}เกิด{\wbr}นี้ 

\section{ไม่{\wbr}มี{\wbr}พระเจ้า{\wbr}ผู้{\wbr}ทรง{\wbr}สร้าง}


นอกจาก{\wbr}นี้ พระพุทธเจ้า{\wbr}ยัง{\wbr}ทรง{\wbr}สอน{\wbr}ด้วย{\wbr}ว่า ไม่{\wbr}มี{\wbr}พระเจ้า หรือ{\wbr}นักบวช หรือ{\wbr}สิ่ง{\wbr}มี{\wbr}ชีวิต{\wbr}ใด ๆ ที่{\wbr}มี{\wbr}อำนาจ{\wbr}แทรกแซง{\wbr}กระบวนการ{\wbr}ทำงาน{\wbr}ของ{\wbr}กรรม{\wbr}ของ{\wbr}บุคคล{\wbr}อื่น{\wbr}ได้  ดังนั้น ศาสนา{\wbr}พุทธ{\wbr}จึง{\wbr}สอน{\wbr}ว่า เรา{\wbr}ทุก{\wbr}คน{\wbr}ต้อง{\wbr}รับผิด{\wbr}และ{\wbr}รับ{\wbr}ชอบ{\wbr}ด้วย{\wbr}ตัวเอง{\wbr}ทั้งหมด  ตัวอย่าง{\wbr}เช่น หาก{\wbr}เรา{\wbr}อยาก{\wbr}รวย เรา{\wbr}ก็{\wbr}ต้อง{\wbr}เอื้อเฟื้อ{\wbr}เผื่อแผ่ ทำ{\wbr}ตัว{\wbr}ให้{\wbr}น่า{\wbr}เชื่อถือ และ{\wbr}ขยัน{\wbr}หมั่น{\wbr}เพียร  และ{\wbr}หาก{\wbr}เรา{\wbr}ต้องการ{\wbr}มี{\wbr}ชีวิต{\wbr}อยู่{\wbr}บน{\wbr}สวรรค์ เรา{\wbr}จะ{\wbr}ต้อง{\wbr}มี{\wbr}ความ{\wbr}เมตตา{\wbr}กรุณา{\wbr}ต่อ{\wbr}ผู้{\wbr}อื่น  ไม่{\wbr}มี{\wbr}พระเจ้า{\wbr}ให้{\wbr}เรา{\wbr}ร้องขอ{\wbr}ความ{\wbr}ช่วยเหลือ หรือ{\wbr}กล่าว{\wbr}อีก{\wbr}นัย{\wbr}หนึ่ง{\wbr}ก็{\wbr}คือ คอร์รัปชัน{\wbr}ใช้{\wbr}ไม่{\wbr}ได้{\wbr}กับ{\wbr}กลไก{\wbr}การ{\wbr}ทำงาน{\wbr}ของ{\wbr}กฎ{\wbr}แห่ง{\wbr}กรรม{\wbr}

ถ้า{\wbr}ถาม{\wbr}ว่า ชาว{\wbr}พุทธ{\wbr}เชื่อ{\wbr}ว่า มี{\wbr}พระเจ้า{\wbr}หรือ{\wbr}พระ{\wbr}ผู้{\wbr}ทรง{\wbr}อำนาจ{\wbr}สูง{\wbr}สุด{\wbr}เป็น{\wbr}ผู้{\wbr}สร้าง{\wbr}จักรวาล{\wbr}หรือ{\wbr}ไม่  ชาว{\wbr}พุทธ{\wbr}จะ{\wbr}ตอบ{\wbr}ด้วย{\wbr}การ{\wbr}ถาม{\wbr}กลับ{\wbr}ว่า คุณ{\wbr}หมาย{\wbr}ถึง{\wbr}จักรวาล{\wbr}ไหน{\wbr}ล่ะ  จักรวาล{\wbr}ที่{\wbr}เรา{\wbr}อยู่{\wbr}ใน{\wbr}ปัจจุบัน{\wbr}นี้{\wbr}ซึ่ง{\wbr}เริ่ม{\wbr}เกิด{\wbr}ขึ้น{\wbr}เมื่อ{\wbr}มี{\wbr}ปรากฏการณ์ ‘บิ{\wbr}ก{\wbr}แบ{\wbr}ง’ (Big Bang) นั้น เป็น{\wbr}แค่{\wbr}หนึ่ง{\wbr}ใน{\wbr}จักรวาล{\wbr}ที่{\wbr}นับ{\wbr}ไม่{\wbr}ถ้วน{\wbr}ตาม{\wbr}คติ{\wbr}พุทธ{\wbr}ที่{\wbr}เกี่ยว{\wbr}กับ{\wbr}โครงสร้าง{\wbr}และ{\wbr}การ{\wbr}กำเนิด{\wbr}ของ{\wbr}จักรวาล  ตาม{\wbr}กฎ{\wbr}ธรรมชาติ{\wbr}ที่{\wbr}ปราศจาก{\wbr}อัตตา{\wbr}และ{\wbr}จุด{\wbr}เริ่มต้น{\wbr}ให้{\wbr}ค้น{\wbr}พบ เมื่อ{\wbr}วัฏจักร{\wbr}ของ{\wbr}จักรวาล{\wbr}หนึ่ง{\wbr}สิ้นสุด{\wbr}ลง จักรวาล{\wbr}ใหม่{\wbr}ก็{\wbr}ถือ{\wbr}กำเนิด{\wbr}ขึ้น วนเวียน{\wbr}เป็น{\wbr}เช่น{\wbr}นี้{\wbr}ไป{\wbr}เรื่อย ๆ  ดังนั้น พระเจ้า{\wbr}ผู้{\wbr}ทรง{\wbr}สร้าง{\wbr}จึง{\wbr}ไม่{\wbr}ได้{\wbr}มี{\wbr}อยู่{\wbr}ใน{\wbr}บริบท{\wbr}นี้{\wbr}

ไม่{\wbr}มี{\wbr}พระเจ้า{\wbr}สูง{\wbr}สุด{\wbr}ที่{\wbr}มา{\wbr}ช่วย{\wbr}ให้{\wbr}เรา{\wbr}รอด{\wbr}พ้น เพราะ{\wbr}เทพ เทวดา มนุษย์ สัตว์{\wbr}และ{\wbr}สิ่ง{\wbr}มี{\wbr}ชีวิต{\wbr}อื่น{\wbr}ต่าง{\wbr}ตก{\wbr}อยู่{\wbr}ภาย{\wbr}ใต้{\wbr}กฎ{\wbr}แห่ง{\wbr}กรรม  แม้{\wbr}กระทั่ง{\wbr}พระพุทธเจ้า{\wbr}เอง{\wbr}ก็{\wbr}ไม่{\wbr}มี{\wbr}พลัง{\wbr}อำนาจ{\wbr}จะ{\wbr}ช่วย{\wbr}ใคร{\wbr}ได้ พระองค์{\wbr}ทรง{\wbr}ทำ{\wbr}ได้{\wbr}เพียง{\wbr}แค่{\wbr}ชี้{\wbr}ให้{\wbr}ผู้{\wbr}มี{\wbr}ปัญญา{\wbr}ได้{\wbr}เห็น{\wbr}ความ{\wbr}จริง{\wbr}ด้วย{\wbr}ตนเอง  เรา{\wbr}ทุก{\wbr}คน{\wbr}ต้อง{\wbr}รับผิดชอบ{\wbr}ต่อ{\wbr}ความ{\wbr}อยู่{\wbr}ดี{\wbr}มี{\wbr}สุข{\wbr}ใน{\wbr}อนาคต{\wbr}ของ{\wbr}ตนเอง  การ{\wbr}ยื่น{\wbr}ความ{\wbr}รับผิดชอบ{\wbr}ให้{\wbr}คน{\wbr}อื่น{\wbr}นั้น{\wbr}เป็น{\wbr}อันตราย{\wbr}อย่าง{\wbr}ยิ่ง{\wbr}

\section{มายาแห่ง ``ตัวตนที่ถาวร"}


พระพุทธเจ้า{\wbr}ทรง{\wbr}สอน{\wbr}ว่า สิ่ง{\wbr}มี{\wbr}ชีวิต{\wbr}ทั้งหลาย{\wbr}นั้น{\wbr}ไม่{\wbr}มี “ตัวตน{\wbr}ที่{\wbr}ถาวร” หรือ{\wbr}แกน{\wbr}หรือ{\wbr}แก่น{\wbr}แท้{\wbr}ถาวร{\wbr}ภายใน{\wbr}ใด ๆ  สิ่ง{\wbr}ที่{\wbr}เรา{\wbr}เรียก{\wbr}ว่า สิ่ง{\wbr}มี{\wbr}ชีวิต ทั้ง{\wbr}มนุษย์{\wbr}และ{\wbr}อื่น ๆ นั้น เป็น{\wbr}เพียง{\wbr}การ{\wbr}รวม{\wbr}ตัว{\wbr}ชั่วคราว{\wbr}ของ{\wbr}ชิ้น{\wbr}ส่วน{\wbr}และ{\wbr}กิจกรรม{\wbr}ต่าง ๆ ซึ่ง{\wbr}เมื่อ{\wbr}ประกอบ{\wbr}กัน{\wbr}สมบูรณ์{\wbr}แล้ว จึง{\wbr}เรียก{\wbr}ว่า “สิ่ง{\wbr}มี{\wbr}ชีวิต”  เมื่อ{\wbr}ชิ้น{\wbr}ส่วน{\wbr}ต่าง ๆ แยก{\wbr}ออก{\wbr}จาก{\wbr}กัน{\wbr}และ{\wbr}กิจกรรม{\wbr}ที่{\wbr}ดำเนิน{\wbr}อยู่ ยุติ{\wbr}ลง สิ่ง{\wbr}นั้น{\wbr}ก็{\wbr}ไม่{\wbr}เรียก{\wbr}ว่า “สิ่ง{\wbr}มี{\wbr}ชีวิต” อีก{\wbr}ต่อ{\wbr}ไป    

เปรียบ{\wbr}ได้{\wbr}กับ{\wbr}คอมพิวเตอร์ ซึ่ง{\wbr}เป็น{\wbr}การ{\wbr}รวม{\wbr}ตัว{\wbr}ของ{\wbr}หลาย{\wbr}ชิ้น{\wbr}ส่วน{\wbr}หลาย{\wbr}กิจกรรม เมื่อ{\wbr}ชิ้น{\wbr}ส่วน{\wbr}ต่าง ๆ ถูก{\wbr}ประกอบ{\wbr}และ{\wbr}ทำงาน{\wbr}ต่าง ๆ ได้{\wbr}อย่าง{\wbr}สมบูรณ์ จึง{\wbr}จะ{\wbr}เรียก{\wbr}ได้{\wbr}ว่า “คอมพิวเตอร์” แต่{\wbr}พอ{\wbr}คอมพิวเตอร์{\wbr}นั้น{\wbr}ถูก{\wbr}แยก{\wbr}ชิ้น{\wbr}ส่วน{\wbr}และ{\wbr}ไม่{\wbr}สามารถ{\wbr}ทำงาน{\wbr}ได้ มัน{\wbr}ก็{\wbr}ไม่{\wbr}ได้{\wbr}เรียก{\wbr}ว่า "คอมพิวเตอร์" อีก{\wbr}ต่อ{\wbr}ไป  ไม่{\wbr}มี{\wbr}แก่น{\wbr}หรือ{\wbr}แกน{\wbr}หลัก{\wbr}อัน{\wbr}ถาวร{\wbr}ใด ๆ ที่{\wbr}เรา{\wbr}จะ{\wbr}เรียก{\wbr}ได้{\wbr}เต็ม{\wbr}ปาก{\wbr}ว่า นี่{\wbr}คือ “คอมพิวเตอร์”  ใน{\wbr}ลักษณะ{\wbr}เดียวกัน สิ่ง{\wbr}มี{\wbr}ชีวิต{\wbr}ก็{\wbr}ไม่{\wbr}มี{\wbr}แก่น{\wbr}หรือ{\wbr}แกน{\wbr}หลัก{\wbr}อัน{\wbr}ถาวร{\wbr}ใด ๆ ที่{\wbr}เรา{\wbr}จะ{\wbr}เรียก{\wbr}ได้{\wbr}ว่า{\wbr}เป็น “ตัวตน{\wbr}ที่{\wbr}ถาวร”

กระนั้น การ{\wbr}เวียน{\wbr}ว่าย{\wbr}ตาย{\wbr}เกิด{\wbr}ก็{\wbr}เป็น{\wbr}ไป{\wbr}ได้{\wbr}แม้{\wbr}ปราศจาก “ตัวตน{\wbr}ที่{\wbr}ถาวร”  ขอ{\wbr}ให้{\wbr}ลอง{\wbr}พิจารณา{\wbr}อุปมา{\wbr}อุปไมย{\wbr}นี้ ใน{\wbr}วัด{\wbr}แห่ง{\wbr}หนึ่ง มี{\wbr}เทียน{\wbr}ไข{\wbr}อยู่{\wbr}เล่ม{\wbr}หนึ่ง{\wbr}ที่{\wbr}กำลัง{\wbr}จะ{\wbr}ดับ พระ{\wbr}รูป{\wbr}หนึ่ง{\wbr}จึง{\wbr}หยิบ{\wbr}เทียน{\wbr}เล่ม{\wbr}ใหม่{\wbr}มา{\wbr}ต่อ{\wbr}กับ{\wbr}เทียน{\wbr}เล่ม{\wbr}เก่า  เทียน{\wbr}เล่ม{\wbr}เก่า{\wbr}ดับ{\wbr}ลง แต่{\wbr}เทียน{\wbr}เล่ม{\wbr}ใหม่{\wbr}ยัง{\wbr}สว่าง{\wbr}ไสว  เทียน{\wbr}เล่ม{\wbr}เก่า{\wbr}ได้{\wbr}ส่ง{\wbr}ทอด{\wbr}อะไร{\wbr}ไป{\wbr}สู่{\wbr}เทียน{\wbr}เล่ม{\wbr}ใหม่{\wbr}หรือ  มี{\wbr}ความ{\wbr}เชื่อม{\wbr}โยง{\wbr}ที่{\wbr}เป็น{\wbr}เหตุ{\wbr}ต่อ{\wbr}เนื่อง{\wbr}กัน แต่{\wbr}ไม่{\wbr}มี “สิ่ง” ใด{\wbr}เดินทาง{\wbr}ข้าม{\wbr}ไป{\wbr}จริง ๆ  โดย{\wbr}นัย{\wbr}เดียวกัน มี{\wbr}ความ{\wbr}เชื่อม{\wbr}โยง{\wbr}ที่{\wbr}เป็น{\wbr}เหตุ{\wbr}ต่อ{\wbr}เนื่อง{\wbr}ระหว่าง{\wbr}ชาติ{\wbr}ก่อน{\wbr}ของ{\wbr}เรา{\wbr}กับ{\wbr}ชาติ{\wbr}ปัจจุบัน แต่{\wbr}ไม่{\wbr}มี “ตัวตน{\wbr}ที่{\wbr}ถาวร” ใด{\wbr}เดินทาง{\wbr}ข้าม{\wbr}จาก{\wbr}ชาติ{\wbr}ก่อน{\wbr}มา{\wbr}ชาติ{\wbr}นี้{\wbr}

อัน{\wbr}ที่{\wbr}จริง{\wbr}แล้ว พระพุทธเจ้า{\wbr}ทรง{\wbr}ตรัส{\wbr}ว่า ความ{\wbr}หลง{\wbr}ใน{\wbr}มายา{\wbr}ว่า{\wbr}มี “ตัวตน{\wbr}ที่{\wbr}ถาวร” เป็น{\wbr}รากเหง้า{\wbr}แห่ง{\wbr}ความ{\wbr}ทุกข์{\wbr}ทั้งมวล{\wbr}ของ{\wbr}มนุษย์ โดย{\wbr}ความ{\wbr}หลง{\wbr}ใน{\wbr}มายา{\wbr}ว่า{\wbr}มี “ตัวตน{\wbr}ที่{\wbr}ถาวร” นี้{\wbr}ปรากฏ{\wbr}ออก{\wbr}มา{\wbr}ใน{\wbr}ลักษณะ “อัตตา” หรือ{\wbr}การ{\wbr}ยึดมั่น{\wbr}ถือ{\wbr}มั่น{\wbr}ใน{\wbr}ตัวตน  ลักษณะ{\wbr}การ{\wbr}ทำงาน{\wbr}ตาม{\wbr}ธรรมชาติ{\wbr}ที่{\wbr}ไม่{\wbr}อาจ{\wbr}หยุด{\wbr}ยั้ง{\wbr}ได้{\wbr}ของ “อัตตา” คือ การ{\wbr}ควบคุม  ถ้า{\wbr}อัตตา{\wbr}เยอะ{\wbr}หรือ{\wbr}มี{\wbr}ความ{\wbr}ยึดมั่น{\wbr}ถือ{\wbr}มั่น{\wbr}ใน{\wbr}ตัวตน{\wbr}สูง ก็{\wbr}จะ{\wbr}ต้องการ{\wbr}ควบคุม{\wbr}ทั้ง{\wbr}โลก  มี{\wbr}อัตตา{\wbr}ปานกลาง{\wbr}ก็{\wbr}อยาก{\wbr}ควบคุม{\wbr}สิ่ง{\wbr}ที่{\wbr}อยู่{\wbr}รอบ{\wbr}ตัว เช่น บ้าน ครอบครัว และ{\wbr}สถานที่{\wbr}ทำงาน  และ{\wbr}ตัว{\wbr}อัตตา{\wbr}ทั้งหลาย{\wbr}นี้{\wbr}ก็{\wbr}พยายาม{\wbr}อย่าง{\wbr}ยิ่ง{\wbr}ที่{\wbr}จะ{\wbr}ควบคุม{\wbr}สิ่ง{\wbr}ที่{\wbr}ตน{\wbr}เข้าใจ{\wbr}ว่า{\wbr}เป็น{\wbr}กาย{\wbr}และ{\wbr}ใจ{\wbr}ของ{\wbr}ตัวเอง{\wbr}

การ{\wbr}ควบคุม{\wbr}นี้{\wbr}แสดงออก{\wbr}มา{\wbr}ใน{\wbr}รูปแบบ{\wbr}ของ{\wbr}ความ{\wbr}อยาก{\wbr}ได้{\wbr}มา{\wbr}และ{\wbr}ความ{\wbr}อยาก{\wbr}ผลัก{\wbr}ออก{\wbr}ไป ส่ง{\wbr}ผล{\wbr}ให้{\wbr}เรา{\wbr}ไม่{\wbr}มี{\wbr}ทั้ง{\wbr}ความ{\wbr}สุข{\wbr}สงบ{\wbr}ภายใน{\wbr}และ{\wbr}ความ{\wbr}สมัคร{\wbr}สมาน{\wbr}กับ{\wbr}คน{\wbr}ภายนอก  และ{\wbr}เจ้า “อัตตา” นี้{\wbr}เอง{\wbr}ที่{\wbr}ผลักดัน{\wbr}ให้{\wbr}เรา{\wbr}เสาะ{\wbr}แสวง{\wbr}หา{\wbr}สิ่ง{\wbr}ต่าง ๆ มา{\wbr}ครอบครอง ชักใย{\wbr}ผู้{\wbr}อื่น และ{\wbr}ใช้{\wbr}หา{\wbr}ประโยชน์{\wbr}จาก{\wbr}สิ่งแวดล้อม{\wbr}อย่าง{\wbr}ไม่{\wbr}ถูกต้อง  เป้าหมาย{\wbr}ของ “อัตตา” ก็{\wbr}เพื่อ{\wbr}ความ{\wbr}สุข{\wbr}ของ{\wbr}ตนเอง แต่{\wbr}กลับ{\wbr}ก่อ{\wbr}ให้{\wbr}เกิด{\wbr}ความ{\wbr}ทุกข์{\wbr}เสมอ  “อัตตา” โหย{\wbr}หา{\wbr}ความ{\wbr}พึงพอใจ แต่{\wbr}กลับ{\wbr}ประสบ{\wbr}แต่{\wbr}ความ{\wbr}ไม่{\wbr}พอใจ  ความ{\wbr}ทุกข์{\wbr}ที่{\wbr}หยั่ง{\wbr}ราก{\wbr}ลึก{\wbr}นี้{\wbr}ไม่{\wbr}มี{\wbr}จุด{\wbr}สิ้นสุด จน{\wbr}กว่า{\wbr}เรา{\wbr}จะ{\wbr}เห็น{\wbr}ได้{\wbr}ด้วย{\wbr}ปัญญา{\wbr}ที่{\wbr}เกิด{\wbr}จาก{\wbr}สมาธิ{\wbr}ลึก{\wbr}ที่{\wbr}มี{\wbr}พลัง{\wbr}ว่า ความคิด{\wbr}ว่า{\wbr}มี “ตัว{\wbr}เรา{\wbr}และ{\wbr}ของ{\wbr}เรา” นั้น{\wbr}เป็น{\wbr}แค่{\wbr}ภาพ{\wbr}มายา{\wbr}

ทั้ง 7 หัวข้อ{\wbr}ที่{\wbr}กล่าว{\wbr}มา{\wbr}นี้ เป็น{\wbr}ตัวอย่าง{\wbr}ของ{\wbr}สิ่ง{\wbr}ที่{\wbr}พระพุทธเจ้า{\wbr}ทรง{\wbr}สอน  และ{\wbr}เพื่อ{\wbr}ให้{\wbr}เห็น{\wbr}ภาพ{\wbr}รวม{\wbr}โดย{\wbr}สังเขป{\wbr}ของ{\wbr}ศาสนา{\wbr}พุทธ{\wbr}อย่าง{\wbr}สมบูรณ์ เรา{\wbr}จะ{\wbr}มา{\wbr}ดู{\wbr}กัน{\wbr}ว่า ปัจจุบัน{\wbr}นี้ มี{\wbr}การ{\wbr}ปฏิบัติ{\wbr}ตาม{\wbr}คำ{\wbr}สอน{\wbr}เหล่า{\wbr}นี้{\wbr}อย่างไร{\wbr}บ้าง{\wbr}

\chapter*{ประเภท{\wbr}ของ{\wbr}พุทธ{\wbr}ศาสนา}
\addcontentsline{toc}{chapter}{ประเภท{\wbr}ของ{\wbr}พุทธ{\wbr}ศาสนา}

อาจ{\wbr}กล่าว{\wbr}ได้{\wbr}ว่า มี{\wbr}พุทธ{\wbr}ศาสนา{\wbr}แบบ{\wbr}เดียว{\wbr}เท่านั้น นั่น{\wbr}คือ คำ{\wbr}สอน{\wbr}ที่{\wbr}พระพุทธเจ้า{\wbr}ตรัส{\wbr}สั่งสอน{\wbr}และ{\wbr}มี{\wbr}การ{\wbr}รวบรวม{\wbr}ไว้{\wbr}เป็น{\wbr}จำนวน{\wbr}มาก  โดย{\wbr}คำ{\wbr}สอน{\wbr}นี้{\wbr}จะ{\wbr}พบ{\wbr}ใน{\wbr}พระ{\wbr}บาลี{\wbr}ปิฎก อัน{\wbr}เป็น{\wbr}คัมภีร์{\wbr}โบราณ{\wbr}ของ{\wbr}พุทธ{\wbr}นิกาย{\wbr}เถรวาท{\wbr}และ{\wbr}เป็น{\wbr}ที่{\wbr}ยอมรับ{\wbr}ทั่วไป{\wbr}ว่า เป็น{\wbr}การ{\wbr}บันทึก{\wbr}คำ{\wbr}ตรัส{\wbr}ของ{\wbr}พระองค์{\wbr}ที่{\wbr}เก่า{\wbr}แก่{\wbr}ที่สุด{\wbr}และ{\wbr}น่า{\wbr}เชื่อถือ{\wbr}มาก{\wbr}ที่สุด โดย{\wbr}นิกาย{\wbr}เถรวาท{\wbr}เป็น{\wbr}ศาสนา{\wbr}สำคัญ{\wbr}ใน{\wbr}ประเทศ{\wbr}ไทย เมียนมา ศรีลังกา กัมพูชา และ{\wbr}ลาว  

ประมาณ 200-300 ปี{\wbr}หลัง{\wbr}จาก{\wbr}พระพุทธเจ้า{\wbr}เสด็จ{\wbr}มหา{\wbr}ปรินิพพาน เหล่า{\wbr}ภิกษุ{\wbr}ภิกษุณี{\wbr}ภาย{\wbr}ใต้{\wbr}คำ{\wbr}สอน{\wbr}ของ{\wbr}พระองค์{\wbr}ก็{\wbr}เริ่ม{\wbr}แบ่งแยก{\wbr}เป็น{\wbr}นิกาย{\wbr}ต่าง ๆ อัน{\wbr}เป็น{\wbr}ผล{\wbr}มา{\wbr}จาก{\wbr}ความ{\wbr}แตก{\wbr}ต่าง{\wbr}ทาง{\wbr}หลักการ{\wbr}และ{\wbr}ระยะ{\wbr}ทาง{\wbr}ที่{\wbr}ห่าง{\wbr}ไกล{\wbr}กัน{\wbr}ทาง{\wbr}ภูมิศาสตร์  นิกาย{\wbr}หนึ่ง{\wbr}คือ นิกาย{\wbr}เถรวาท ซึ่ง{\wbr}เดิม{\wbr}มี{\wbr}รากฐาน{\wbr}ใน{\wbr}ศรีลังกา{\wbr}เป็น{\wbr}หลัก  อีก{\wbr}นิกาย{\wbr}คือ นิกาย{\wbr}สร{\wbr}วา{\wbr}สติ{\wbr}วาท ซึ่ง{\wbr}มี{\wbr}ผู้{\wbr}นับถือ{\wbr}มาก{\wbr}ใน{\wbr}แคว้น{\wbr}กัษ{\wbr}มี{\wbr}ระ{\wbr}หรือ{\wbr}แคชเมียร์  คัมภีร์{\wbr}ของ{\wbr}นิกาย{\wbr}สร{\wbr}วา{\wbr}สติ{\wbr}วาท{\wbr}จำนวน{\wbr}มาก{\wbr}ได้{\wbr}รับ{\wbr}การ{\wbr}แปล{\wbr}เป็น{\wbr}ภาษา{\wbr}จีน และ{\wbr}ใน{\wbr}ปัจจุบัน{\wbr}ก็{\wbr}ยัง{\wbr}สามารถ{\wbr}พบ{\wbr}อ่าน{\wbr}ได้  โดย{\wbr}รวม ใน{\wbr}ช่วง{\wbr}ศตวรรษ{\wbr}แรก ๆ ภายหลัง{\wbr}พระพุทธเจ้า{\wbr}เสด็จ{\wbr}มหา{\wbr}ปรินิพพาน พุทธ{\wbr}ศาสนา{\wbr}มี{\wbr}ประมาณ 20 นิกาย{\wbr}กระจาย{\wbr}ตาม{\wbr}แว่นแคว้น{\wbr}ต่าง ๆ ของ{\wbr}อินเดีย 

ใน{\wbr}ช่วง{\wbr}ต้น{\wbr}คริสตกาล มี{\wbr}คัมภีร์{\wbr}ที่{\wbr}เดิม{\wbr}ไม่{\wbr}เป็น{\wbr}ที่{\wbr}รู้จัก{\wbr}ปรากฏ{\wbr}ขึ้น โดย{\wbr}มี{\wbr}ความ{\wbr}พยายาม{\wbr}ให้{\wbr}เหตุผล{\wbr}ว่า พระ{\wbr}โพธิสัตว์{\wbr}ยิ่งใหญ่{\wbr}เหนือ{\wbr}พระ{\wbr}อรหันต์  แนว{\wbr}ความ{\wbr}เคลื่อนไหว{\wbr}ครั้ง{\wbr}ใหม่{\wbr}นี้{\wbr}เรียก{\wbr}ตนเอง{\wbr}ว่า มหายาน  ชาว{\wbr}มหายาน{\wbr}ยัง{\wbr}คง{\wbr}รักษา{\wbr}คำ{\wbr}สอน{\wbr}ดั้งเดิม{\wbr}ของ{\wbr}พระพุทธเจ้า (ซึ่ง{\wbr}เป็น{\wbr}ที่{\wbr}รู้จัก{\wbr}กัน{\wbr}ใน{\wbr}นาม “คัมภีร์{\wbr}อา{\wbr}ค{\wbr}มะ” หรือ “นิ{\wbr}กา{\wbr}ยะ”) แต่{\wbr}ภายหลัง ชาว{\wbr}มหายาน{\wbr}นับถือ{\wbr}คำ{\wbr}สอน{\wbr}ดั้งเดิม{\wbr}นี้{\wbr}เป็นรอง{\wbr}จาก{\wbr}การ{\wbr}ตีความ{\wbr}และ{\wbr}แนวคิด{\wbr}ใหม่ ๆ ที่{\wbr}มี{\wbr}ใน{\wbr}คัมภีร์{\wbr}มหายาน{\wbr}

พุทธ{\wbr}ศาสนา{\wbr}ที่{\wbr}ได้{\wbr}ประดิษฐาน{\wbr}ใน{\wbr}ประเทศ{\wbr}จีน{\wbr}และ{\wbr}ยัง{\wbr}คง{\wbr}รุ่งเรือง{\wbr}อยู่{\wbr}ใน{\wbr}ไต้หวัน{\wbr}ใน{\wbr}ปัจจุบัน{\wbr}นั้น{\wbr}แสดง{\wbr}ถึง{\wbr}พัฒนาการ{\wbr}ช่วง{\wbr}แรก{\wbr}ของ{\wbr}นิกาย{\wbr}มหายาน  และ{\wbr}จาก{\wbr}ประเทศ{\wbr}จีน นิกาย{\wbr}มหายาน{\wbr}ก็{\wbr}แพร่หลาย{\wbr}ไป{\wbr}ใน{\wbr}เวียดนาม เกาหลี{\wbr}และ{\wbr}ญี่ปุ่น ผล{\wbr}จาก{\wbr}การ{\wbr}แผ่{\wbr}ขยาย{\wbr}นี้{\wbr}คือ กำเนิด{\wbr}ของ{\wbr}นิกาย{\wbr}เซน  ส่วน{\wbr}ศาสนา{\wbr}พุทธ{\wbr}ใน{\wbr}ทิเบต{\wbr}และ{\wbr}มองโกเลีย{\wbr}เป็น{\wbr}พัฒนาการ{\wbr}ที่{\wbr}เกิด{\wbr}ขึ้น{\wbr}ใน{\wbr}ช่วง{\wbr}หลัง โดย{\wbr}มัก{\wbr}เรียก{\wbr}ว่า “นิกาย{\wbr}วัชรยาน”  

\chapter*{ความ{\wbr}สัมพันธ์{\wbr}ของ{\wbr}ศาสนา{\wbr}พุทธ{\wbr}กับ{\wbr}โลก{\wbr}ยุค{\wbr}ปัจจุบัน}
\addcontentsline{toc}{chapter}{ความ{\wbr}สัมพันธ์{\wbr}ของ{\wbr}ศาสนา{\wbr}พุทธ{\wbr}กับ{\wbr}โลก{\wbr}ยุค{\wbr}ปัจจุบัน}

ปัจจุบัน พุทธ{\wbr}ศาสนา{\wbr}ยัง{\wbr}คง{\wbr}เป็น{\wbr}ที่{\wbr}ยอมรับ{\wbr}นับถือ{\wbr}ใน{\wbr}หลาย{\wbr}ประเทศ{\wbr}นอกเหนือ{\wbr}ถิ่น{\wbr}กำเนิด{\wbr}เดิม  ผู้คน{\wbr}ทั่ว{\wbr}โลก{\wbr}จำนวน{\wbr}มาก{\wbr}พิจารณา{\wbr}เลือก{\wbr}รับ{\wbr}แนว{\wbr}ปฏิบัติ{\wbr}แบบ{\wbr}พุทธ{\wbr}ที่{\wbr}สงบ{\wbr}ร่มเย็น เปี่ยม{\wbr}เมตตา{\wbr}กรุณา และ{\wbr}มี{\wbr}ความ{\wbr}รับผิดชอบ{\wbr}

คำ{\wbr}สอน{\wbr}ของ{\wbr}พระพุทธเจ้า{\wbr}เรื่อง{\wbr}กฎ{\wbr}แห่ง{\wbr}กรรม{\wbr}ทำให้{\wbr}มนุษย์{\wbr}มี{\wbr}ทั้ง{\wbr}พื้นฐาน{\wbr}อัน{\wbr}เที่ยงธรรม{\wbr}ไร้{\wbr}มลทิน{\wbr}และ{\wbr}เหตุผล{\wbr}สำหรับ{\wbr}การ{\wbr}ใช้{\wbr}ชีวิต{\wbr}ตาม{\wbr}ทำนอง{\wbr}คลอง{\wbr}ธรรม  ดังนั้น หาก{\wbr}บ้านเมือง{\wbr}ใด{\wbr}ยอมรับ{\wbr}กฎ{\wbr}แห่ง{\wbr}กรรม{\wbr}นี้{\wbr}อย่าง{\wbr}กว้าง{\wbr}ขวาง{\wbr}ก็{\wbr}จะ{\wbr}นำพา{\wbr}ไป{\wbr}สู่{\wbr}สังคม{\wbr}ที่{\wbr}เข้มแข็ง เอื้อ{\wbr}อาทร และ{\wbr}เปี่ยม{\wbr}ด้วย{\wbr}คุณ{\wbr}งาม{\wbr}ความ{\wbr}ดี{\wbr}มาก{\wbr}ยิ่ง{\wbr}ขึ้น 

คำ{\wbr}สอน{\wbr}เกี่ยว{\wbr}กับ{\wbr}การ{\wbr}เวียน{\wbr}ว่าย{\wbr}ตาย{\wbr}เกิด{\wbr}ทำให้{\wbr}เรา{\wbr}มี{\wbr}มุมมอง{\wbr}ที่{\wbr}กว้าง{\wbr}ขึ้น{\wbr}เกี่ยว{\wbr}กับ{\wbr}ชีวิต{\wbr}อัน{\wbr}แสน{\wbr}สั้น{\wbr}ใน{\wbr}ชาติ{\wbr}นี้ ทำให้{\wbr}เรา{\wbr}ให้{\wbr}ความหมาย{\wbr}ของ{\wbr}การ{\wbr}เกิด{\wbr}และ{\wbr}การ{\wbr}ตาย{\wbr}มาก{\wbr}ยิ่ง{\wbr}ขึ้น  ความ{\wbr}เข้าใจ{\wbr}เรื่อง{\wbr}การ{\wbr}เวียน{\wbr}ว่าย{\wbr}ตาย{\wbr}เกิด{\wbr}ช่วย{\wbr}ลด{\wbr}ความ{\wbr}เศร้าโศก{\wbr}และ{\wbr}สลด{\wbr}อาลัย{\wbr}ต่อ{\wbr}ความ{\wbr}ตาย และ{\wbr}ช่วย{\wbr}ให้{\wbr}เรา{\wbr}หัน{\wbr}ไป{\wbr}ใส่ใจ{\wbr}กับ{\wbr}คุณภาพ{\wbr}ของ{\wbr}การ{\wbr}ใช้{\wbr}ชีวิต{\wbr}มาก{\wbr}กว่า{\wbr}ความ{\wbr}ยืน{\wbr}ยาว{\wbr}ของ{\wbr}ชีวิต{\wbr}

นับ{\wbr}แต่{\wbr}แรก{\wbr}เริ่ม การ{\wbr}เจริญ{\wbr}จิตตภาวนา{\wbr}เป็น{\wbr}หัวใจ{\wbr}สำคัญ{\wbr}ของ{\wbr}แนว{\wbr}ปฏิบัติ{\wbr}แบบ{\wbr}พุทธ  ปัจจุบัน การ{\wbr}ทำ{\wbr}สมาธิ{\wbr}ได้{\wbr}รับ{\wbr}ความ{\wbr}นิยม{\wbr}เพิ่ม{\wbr}มาก{\wbr}ขึ้น{\wbr}เรื่อย ๆ เนื่อง{\wbr}จาก{\wbr}มี{\wbr}ผล{\wbr}พิสูจน์{\wbr}เป็น{\wbr}ที่{\wbr}ทราบ{\wbr}แล้ว{\wbr}ว่า เป็น{\wbr}ประโยชน์{\wbr}ต่อ{\wbr}ทั้ง{\wbr}สุขภาพ{\wbr}กาย{\wbr}และ{\wbr}สุขภาพ{\wbr}ใจ  เมื่อ{\wbr}เห็น{\wbr}แล้ว{\wbr}ว่า ความ{\wbr}เครียด{\wbr}เป็น{\wbr}หนึ่ง{\wbr}ใน{\wbr}สาเหตุ{\wbr}หลัก{\wbr}ของ{\wbr}ความ{\wbr}ทุกข์ การ{\wbr}เจริญ{\wbr}จิตตภาวนา{\wbr}ที่{\wbr}เสริมสร้าง{\wbr}ความ{\wbr}สงบ{\wbr}เย็น{\wbr}ก็{\wbr}ยิ่ง{\wbr}กลาย{\wbr}เป็น{\wbr}สิ่ง{\wbr}ที่{\wbr}มี{\wbr}คุณค่า{\wbr}มาก{\wbr}ขึ้น  

โลก{\wbr}ของ{\wbr}เรา{\wbr}นี้{\wbr}ทั้ง{\wbr}เล็ก{\wbr}และ{\wbr}เปราะบาง{\wbr}เกิน{\wbr}กว่า{\wbr}ที่{\wbr}พวก{\wbr}เรา{\wbr}จะ{\wbr}ใช้{\wbr}ชีวิต{\wbr}ด้วย{\wbr}ความ{\wbr}โกรธ{\wbr}เคือง{\wbr}และ{\wbr}เพียง{\wbr}ลำพัง  ดังนั้น การ{\wbr}เปิด{\wbr}ใจ{\wbr}กว้าง{\wbr}และ{\wbr}ความ{\wbr}เมตตา{\wbr}กรุณา{\wbr}จึง{\wbr}มี{\wbr}ความ{\wbr}สำคัญ{\wbr}เป็น{\wbr}อย่าง{\wbr}ยิ่ง  คุณสมบัติ{\wbr}ดัง{\wbr}กล่าว{\wbr}นี้{\wbr}เป็น{\wbr}ปัจจัย{\wbr}หลัก{\wbr}ที่{\wbr}ก่อ{\wbr}ให้{\wbr}เกิด{\wbr}ความ{\wbr}สุข เป็น{\wbr}สิ่ง{\wbr}ที่{\wbr}เรา{\wbr}สามารถ{\wbr}พัฒนา{\wbr}ได้{\wbr}โดย{\wbr}การ{\wbr}เจริญ{\wbr}จิตตภาวนา{\wbr}ตาม{\wbr}แนว{\wbr}พุทธ{\wbr}ศาสนา{\wbr}และ{\wbr}ยัง{\wbr}เป็น{\wbr}สิ่ง{\wbr}ที่{\wbr}เรา{\wbr}ควร{\wbr}เพียร{\wbr}นำ{\wbr}ไป{\wbr}ปฏิบัติใช้ใน{\wbr}ชีวิต{\wbr}ประจำ{\wbr}วัน  

การ{\wbr}ให้{\wbr}อภัย ความ{\wbr}อ่อนโยน ความ{\wbr}ไม่{\wbr}เป็น{\wbr}พิษ{\wbr}เป็น{\wbr}ภัย และ{\wbr}ความ{\wbr}เมตตา{\wbr}กรุณา{\wbr}อัน{\wbr}สงบ{\wbr}เย็น{\wbr}เป็น{\wbr}คุณลักษณะ{\wbr}เด่น{\wbr}ของ{\wbr}พุทธ{\wbr}ศาสนา{\wbr}ซึ่ง{\wbr}เป็น{\wbr}ที่{\wbr}ทราบ{\wbr}กัน{\wbr}ดี และ{\wbr}เป็น{\wbr}สิ่ง{\wbr}ที่{\wbr}มอบ{\wbr}ให้{\wbr}กับ{\wbr}สรรพ{\wbr}สัตว์{\wbr}ทั้งหลาย{\wbr}อย่าง{\wbr}ไม่{\wbr}มี{\wbr}เงื่อนไข ซึ่ง{\wbr}นับ{\wbr}รวม{\wbr}ไป{\wbr}ถึง{\wbr}สัตว์{\wbr}เดรัจฉาน และ{\wbr}ที่{\wbr}สำคัญ{\wbr}ที่สุด{\wbr}ก็{\wbr}คือ ตัว{\wbr}ของ{\wbr}เรา{\wbr}เอง  ศาสนา{\wbr}พุทธ{\wbr}ไม่{\wbr}มี{\wbr}พื้นที่{\wbr}ให้{\wbr}กับ{\wbr}การ{\wbr}จม{\wbr}อยู่{\wbr}กับ{\wbr}ความ{\wbr}รู้สึก{\wbr}ผิด{\wbr}หรือ{\wbr}การ{\wbr}ชิงชัง{\wbr}ตนเอง ไม่{\wbr}มี{\wbr}แม้{\wbr}กระทั่ง{\wbr}พื้นที่{\wbr}ให้{\wbr}กับ{\wbr}ความ{\wbr}รู้สึก{\wbr}ผิด{\wbr}ที่{\wbr}เรา{\wbr}รู้สึก{\wbr}ผิด{\wbr}อีก{\wbr}ด้วย!

คำ{\wbr}สอน{\wbr}และ{\wbr}หลัก{\wbr}ปฏิบัติ{\wbr}ตาม{\wbr}แนวทาง{\wbr}ดัง{\wbr}กล่าว{\wbr}ข้าง{\wbr}ต้น นำ{\wbr}มา{\wbr}ซึ่ง{\wbr}ความ{\wbr}เมตตา{\wbr}ปรานี{\wbr}ที่{\wbr}อ่อนโยน ความ{\wbr}สุข{\wbr}สงบ{\wbr}อัน{\wbr}ไม่{\wbr}สั่น{\wbr}คลอน และ{\wbr}ปัญญา อัน{\wbr}เป็น{\wbr}คุณลักษณะ{\wbr}ที่{\wbr}มอง{\wbr}ว่า{\wbr}เป็น{\wbr}เอกลักษณ์{\wbr}ของ{\wbr}พุทธ{\wbr}ศาสนา{\wbr}มา{\wbr}ตลอด{\wbr}ระยะ{\wbr}เวลา{\wbr}กว่า 25 ศตวรรษ และ{\wbr}เป็น{\wbr}สิ่ง{\wbr}ที่{\wbr}จำเป็น{\wbr}อย่าง{\wbr}ยิ่ง{\wbr}สำหรับ{\wbr}โลก{\wbr}เรา{\wbr}ใน{\wbr}ปัจจุบัน  

ความ{\wbr}สุข{\wbr}สงบ{\wbr}และ{\wbr}การ{\wbr}เปิด{\wbr}ใจ{\wbr}กว้าง{\wbr}อัน{\wbr}เกิด{\wbr}จาก{\wbr}ปรัชญา{\wbr}ความคิด{\wbr}อัน{\wbr}ลึกซึ้ง{\wbr}และ{\wbr}สม{\wbr}เหตุ{\wbr}สม{\wbr}ผล{\wbr}จึง{\wbr}เป็น{\wbr}สิ่ง{\wbr}ที่{\wbr}ทำให้{\wbr}คำ{\wbr}สอน{\wbr}ของ{\wbr}พระพุทธเจ้า{\wbr}มี{\wbr}คุณ{\wbr}ประโยชน์{\wbr}มหาศาล{\wbr}ต่อ{\wbr}ชีวิต{\wbr}ของ{\wbr}ผู้คน{\wbr}เสมอ{\wbr}มา{\wbr}โดย{\wbr}ไม่{\wbr}จำกัด{\wbr}กาล{\wbr}เวลา 



\end{document}